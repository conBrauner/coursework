\documentclass[11pt, letterpaper]{article}
\usepackage[margin=1.5cm]{geometry}
\pagestyle{plain}

\usepackage{amsmath, amsfonts, amssymb, amsthm}
\usepackage[shortlabels]{enumitem}
\usepackage[makeroom]{cancel}
\usepackage{graphicx}
\graphicspath{{./images/}}

\begin{document}
\title{Assignment 3\\\normalsize MATH391}
\author{Connor Braun}

\allowdisplaybreaks

\maketitle
\section*{Problem 1}
\subsection*{a) \normalfont Function newton\_bisect() for numerically approximating root of some $f(x)=0$
bracketed by some interval $[a,b]$. The function combines the bisection and Newton's methods. Function specifications
are as described in the assignment.}
\begin{verbatim}
    function [x, fx, nf, af, bf] = newton_bisect(a, b, fname, tolx, nmax)

        % Fix arguments pro re nada
        if a > b
            a_c = a;
            a = b;
            b = a_c;
        end
        tolx = abs(tolx);
        nmax = round(abs(nmax));
        fname_p = matlabFunction(diff(sym(fname)));
        
        % Variable initialization
        x = a;
        nf = 2; % Two function evaluations during initialization
        af = a;
        bf = b;

        % Evaluate functions at endpoints  
        fa = fname(a);
        fb = fname(b);
        fx = fa;
        
        % Return condition: either initial endpoint is a root
        if fa == 0 
            disp('newton_bisect return condition: a is a root')
            return
        elseif fb == 0 
            x = b;
            disp('newton_bisect return condition: b is a root')
            return
        else % Assert the guaranteed existence of a root on (a,b)
            assert(fa*fb <= 0, 'Root may not exist; ensure that the product of function evaluated at interval endpoints is negative')
        end
        
        % Root finding algorithm main loop
        for i = 1:nmax
        
            % Return condition: interval length less than allowed tolerance
            if abs(b - a) <= tolx
                disp('newton_bisect return condition: |b - a| < tolx')
                return
            end
            
            fx_p = fname_p(x); % Compute derivative
            
            % First, try iterating using Newton's algorithm
            x = x - fx/fx_p; 
            fx = fname(x); 
            
            if x >= a && x <= b % If the next iterate is on the current interval
                if fx*fa < 0 % If (a,x) brackets the root
                    b = x; 
                else % Otherwise, (x,b) must bracket the root
                    a = x;
                end
            
            % Next, if the iterate left the interval, resort to bisection
            else 
                x = (a + b)/2;
                fx = fname(x);
                
                if fx*fa < 0 % If (a,x) brackets the root
                    b = x;
                elseif fx*fb < 0 % If (x,b) brackets the root
                    a = x;
                end
            end
            fa = fname(a);
            fb = fname(b);
            af = a;
            bf = b;
            nf = nf + (i - 1)*5; % 5 function evaluations per iteration
        end
    end
\end{verbatim}
\section*{Problem 2}
\subsection*{a) \normalfont Let $f(x)=x^3-a$ for some $a\in\mathbb{R}$ and consider the equation $x^3-a=0$. 
Write down the secant algorithm for $f(x)=0$.}
Let $x_1=b$ and $x_2=c$ be real numbers. The secant algorithm generates a sequence of iterates $\{x_n\}_{n\geq1}$
approximating $r\in\mathbb{R}$ such that $f(r)=0$, where $r$ is called a root of $f(x)$. The $n^{th}$ iterate $x_n$
where $x \geq 3$ is given by
\begin{align*}
    x_n=x_{n-1}-f(x_{n-1})\frac{x_{n-1}-x_{n-2}}{f(x_{n-1})-f(x_{n-2})}.\\
\end{align*}
We can simplify this equation as follows
\begin{align*}
    x_n&=x_{n-1}-(x_{n-1}^3-a)\frac{x_{n-1} - x_{n-2}}{(x_{n-1}^3-a) - (x_{n-2}^3-a)}\\
    &=x_{n-1}-\frac{(x_{n-1}^3-a)(x_{n-1} - x_{n-2})}{x_{n-1}^3 - x_{n-2}^3}\\
    &=\frac{x_{n-1}(x_{n-1}^3-x_{n-2}^3) - (x_{n-1}^3-a)(x_{n-1} - x_{n-2})}{x_{n-1}^3 - x_{n-2}^3}.
\end{align*}
However, $x_{n-1}^3-x_{n-2}^3=(x_{n-1}-x_{n-2})(x_{n-1}^2+x_{n-1}x_{n-2}+x_{n-2}^2)$, so we can expand
the expression above into
\begin{align*}
    x_n&=\frac{x_{n-1}(x_{n-1}-x_{n-2})(x_{n-1}^2+x_{n-1}x_{n-2}+x_{n-2}^2)-(x_{n-1}^3-a)(x_{n-1}-x_{n-2})}{(x_{n-1}-x_{n-2})(x_{n-1}^2+x_{n-1}x_{n-2}+x_{n-2}^2)}\\
    &=\frac{x_{n-1}(x_{n-1}^2+x_{n-1}x_{n-2}+x_{n-2}^2)-x_{n-1}^3+a}{x_{n-1}^2+x_{n-1}x_{n-2}+x_{n-2}^2}\\
    &=\frac{x_{n-1}^3+x_{n-1}^2x_{n-2}+x_{n-2}^2x_{n-1}-x_{n-1}^3+a}{x_{n-1}^2+x_{n-1}x_{n-2}+x_{n-2}^2}\\
    &=\frac{x_{n-1}^2x_{n-2}+x_{n-2}^2x_{n-1}+a}{x_{n-1}^2+x_{n-1}x_{n-2}+x_{n-2}^2}\\
    &=\frac{x_{n-1}x_{n-2}(x_{n-2}+x_{n-1})+a}{x_{n-1}^2+x_{n-1}x_{n-2}+x_{n-2}^2}.
\end{align*}
Finally, we can write the secant algorithm for $x^3-a=0$ as\\
\begin{align*}
    \begin{cases}
        x_1=b\\
        x_2=c\\
        x_n=\frac{x_{n-1}x_{n-2}(x_{n-2}+x_{n-1})+a}{x_{n-1}^2+x_{n-1}x_{n-2}+x_{n-2}^2},\text{\indent $n\geq 3$}.
    \end{cases}
\end{align*}
\newpage
\subsection*{b) \normalfont Let $a=3$, $b=1$ and $c=2$, then use the secant algorithm defined in a)
to compute the first $10$ iterates. Comment on the rate of convergence of the sequence.}
{\bf Table 1.}
The sequence of approximations of $r$ such that $f(r)=r^3-3=0$ generated by the secant algorithm.
The computation for $x_3$ using the result in a) is shown, and the rest were computed to $19$ digits using the
MATLAB code shown below the table. We know that $r=3^{\frac{1}{3}}=1.442249570307408382\dots$, and so also tabulate $k_n$,
where $k_n$ is the number of consecutive correct digits in $x_n$.
\begin{table}[h!]
    \begin{center}
        \begin{tabular}{c|c|c}
            $n$ & $x_n$ & $k_n$\\
            \hline
            $1$ & $1$ & -\\
            $2$ & $2$ & -\\
            $3$ & $x_3=\frac{2(3)+3}{4+2+1}=\frac{9}{7}=1.285714285714285714\dots$ & 1\\
            $4$ & $1.392059553349875931\dots$ & 1\\
            $5$ & $1.448265350468348823\dots$ & 3\\
            $6$ & $1.442035892099417439\dots$ & 4\\
            $7$ & $1.442248681418064631\dots$ & 6\\
            $8$ & $1.442249570439115908\dots$ & 10\\
            $9$ & $1.442249570307408301\dots$ & 17\\
            $10$ & $1.442249570307408382\dots$ & 19\\
            $11$ & $1.442249570307408382\dots$ & 19\\
            $12$ & $1.442249570307408382\dots$ & 19
        \end{tabular}
    \end{center}
\end{table}\\
MATLAB code to generate results:
\begin{verbatim}
    digits(19)
    a = 3;
    xn = [vpa(1); vpa(2)];
    for i = numel(xn) + 1:numel(xn) + 10
        xi = (xn(i - 1)*xn(i - 2)*(xn(i - 2) + xn(i - 1)) + a)/...
            (xn(i - 1)^2 + xn(i - 1)*xn(i - 2) + xn(i - 2)^2);
        xn(i) = xi;
    end
    xn
\end{verbatim}
The sequence appears to have a faster than linear rate of convergence, since the number of correct
decimal places increases nonlinearly across the tabulated sequence of iterations. We next endeavor to 
approximate the rate of convergence.
To motivate the approximation, first suppose that $p$ is the rate of convergence of the sequence $\{x_n\}_{n\geq 1}$ with $x_1=b$ and
$x_2=c$ and $x_n$, $n\geq 3$ as defined in part a). Then there exists some $d>0$ such that
\begin{align*}
    \lim_{n \to +\infty} \frac{|x_{n+1}-r|}{|x_n-r|^p}&=d\\
    \Rightarrow\frac{|x_{n+1}-r|}{|x_n-r|^p}&\approx d
\end{align*}
We can now manipulate this expression to reveal an approximation for $p$, which we will call $p_n$ where $n\geq 3$.
\begin{align*}
    |x_{n+1}-r|&\approx d|x_n-r|^p\\
    \Rightarrow\frac{|x_{n+1}-r|}{|x_n-r|}&\approx \frac{d|x_n-r|^p}{d|x_{n-1}-r|^p}\\
    \Rightarrow\ln\left(\frac{|x_{n+1}-r|}{|x_n-r|}\right)&\approx \ln\left(\frac{|x_n-r|^p}{|x_{n-1}-r|^p}\right)\\
    \Rightarrow p_{n+1}=\frac{\ln\left(\frac{|x_{n+1}-r|}{|x_n-r|}\right)}{\ln\left(\frac{|x_n-r|}{|x_{n-1}-r|}\right)}&\approx p
\end{align*}
This approximation, while true, is numerically unstable since, as seen in table 1, the difference between
subsequent iterates rapidly approaches zero, which results in division by zero when precision is limited. To avoid this, we most seriously consider
$5\leq n \leq 8$, since the detectable difference between iterates diminishes beyond this point (indeed, we gain
7 correct decimal places for $n=9$, 2 for $n=10$, and 0 thereafter). \\\\
{\bf Table 2.} 
Approximated values of $p$ using the last 3 elements of $\{x_i\}$ for $i\leq n$, where sequence $\{x_i\}$ is
generated using the secant algorithm as defined in a) to approximate root $r$ of $x^3-3=0$.  
\begin{table}[h!]
    \begin{center}
        \begin{tabular}{c|c}
            $n$ & $p_n$\\
            \hline
            $5$ & $1.86505038093259$\\
            $6$ & $1.57331139189452$\\
            $7$ & $1.64253909945344$\\
            $8$ & $1.60830639153794$\\
            \hline
            $9$ & $2.17371661563057$\\
            $10$ & $-0.25350888214223$\\
            $11$ & $8.25611634512994\times 10^{-12}$\\
            $12$ & $-1.00000000004011$\\
            $13$ & $0$\\
            $14$ & NaN
        \end{tabular}
    \end{center}
\end{table}\\
For $n<9$ (i.e., before the onset of numerical instability as described above) we have that $p\approx 1.6 > 1$, 
so the rate of convergence of the sequence $\{x_n\}_{n\geq 1}$ generated by the secant algorithm defined in 
a) to find $r$ for $a=3$ appears to be superlinear as per the finding that we appear to have $1<p<2$. 
\section*{Problem 3}
\subsection*{a) \normalfont Prove that if $g(x)\in C^0[a,b]$ and if $\forall_x\in[a,b]$, $a\leq g(x)\leq b$,
then $g(x)$ has at least on fixed point in $[a,b]$. Justify whether or not the fixed point is necessarily unique.}
{\bf Proof.} Suppose that $g(x)$ is a continuous function on $[a,b]$ for some $a,b\in\mathbb{R}$. Suppose also that 
$\forall_x\in[a,b]$, we have that $a\leq g(x)\leq b$. First, we define a new function $f(x)=g(x)-x$, which is
itself continuous on $[a,b]$ since both $g(x)$ and $x$ are. Next, we can show that $f(x)$ has a root $r$ on $[a,b]$.
\begin{align*}
    &a\leq g(a)\leq b, \text{\indent since $a\leq a\leq b$}\\
    &\Rightarrow 0\leq g(a)-a\leq b-a\\
    &\Rightarrow 0\leq f(a).
\end{align*}
Furthermore,
\begin{align*}
    &a\leq g(b)\leq b, \text{\indent since $a\leq b\leq b$}\\
    &\Rightarrow a-b\leq g(b)-b\leq 0\\
    &\Rightarrow f(b)\leq 0.
\end{align*}
Now, since $f(x)$ is continuous on $[a,b]$ and $f(b)\leq 0\leq f(a)$, by the intermediate value theorem
we guarantee the existence of some $\xi\in[a,b]$ such that $f(\xi)=0$. Let $r$ be this value so that $f(r)=0$.
Then $r$ is a fixed point of $g(x)$, since
\begin{align*}
    f(r)&=g(r)-r\\
    \Rightarrow g(r)&=r
\end{align*}
where $r$ is guaranteed to exist under the conditions provided. \\\\
Such a fixed point need not be unique, however. \\\\
{\bf Proof.} Let $g(x)=x$, $a=0$, $b=1$ and $c\in[a,b]$. Then $g(x)\in C^0[a,b]$ and $a\leq c\leq b\Rightarrow a\leq g(c)\leq b$.
Despite this, we have both that $g(0)=0$ and $g(1)=1$, so both $0,1\in [a,b]$ are fixed points of $g(x)$.
\section*{Problem 4}
\subsection*{a) \normalfont Consider the equation $f(x)=0$, where $f(x)=e^x-ex$. Show that $f(x)$ has a unique
root $r$ and find it's multiplicity $p$.}
First we seek the critical points of $f(x)$.
\begin{align*}
    f'(x)&=e^x-e=0\\
    \Rightarrow e^x&=e\\
    \Rightarrow x\ln(e)&=ln(e)\\
    \Rightarrow x&=1.
\end{align*}
Hence $x=1$ is the only critical point of $f(x)$, so computing $f'(x_a)$ for $x_a=1+1=2>1$ and $f'(x_b)$ for $x_b=1-1=0<1$ shows us that
\begin{align*}
    f'(x_a)=f'(2)&= e^2-e=e(e-1)>0\\
    f'(x_b)=f'(0)&= e^0-e=1-e<0.
\end{align*}
From this, $f(x)$ is decreasing for all $x<1$, and increasing for all $x>1$. Furthermore, $f(1)=e^1-e=0$, so
$r=1$ is the only root of $f(x)=0$. Now, since $r=1$ is a root of $f(x)$, for some unknown function $h(x)$ we can write
\begin{align*}
    f(x)&=(x-1)h(x)\\
    \Rightarrow f'(x)&=h(x)+(x-1)h'(x)\\
    \Rightarrow f'(1)&=e^1-e=0=h(1)
\end{align*} 
where now $r=1$ is also a root of $h(x)$. Then for some function $g(x)$ we can write
\begin{align*}
    f(x)&=(x-1)^2g(x)\\
    \Rightarrow f'(x)&=2(x-1)g(x)+(x-1)^2g'(x)\\
    \Rightarrow f''(x)&=2g(x)+4(x-1)g'(x)+(x-1)^2g''(x)\\
    \Rightarrow f''(1)&=e^1=2g(1)\\
    \Rightarrow g(1)&=e/2\neq 0
\end{align*}
So 1 is not a root of $g(x)=0$ and we can write that
\begin{align*}
    f(x)=(x-1)^2g(x)
\end{align*}
implying that root $r=1$ of $f(x)=0$ has multiplicity 2.
\newpage
\subsection*{b) \normalfont Let $\{x_n\}_{n\geq 1}$ be the Newton sequence with initial guess $x_1=0$. Compute 
$20$ iterates and and verify the linear convergence of the sequence.}
{\bf Table 3.} First 20 iterates of the Newton sequence as described above. The error $e_n=|x_n-1|$ and 
approximation of the rate of convergence $k_n$ are also tabulated, where the latter uses the same approximation technique
as was used to compute $p_n$ in table 2. Note that the $n^{th}$ approximation of the rate of convergence, $k_n$ uses error of iterates $n$, $n-1$ and $n-2$. 
All calculations were done in MATLAB with 15 points of precision. The code used to generate 
this data is included below the table. 
\begin{table}[h!]
    \begin{center}
        \begin{tabular}{c|c|c|c}
            $n$ & $x_n$ & $e_n$ & $k_n$\\
            \hline
            $1$ & $0$ & 1 & $-$\\
            $2$ & $0.5819767068693262$ & $0.418023293130674$ & $-$\\
            $3$ & $0.805508075137019$ & $0.194491924862981$ & $0.877242161526464$\\
            $4$ & $0.905904311079007$ & $0.0940956889209927$ & $0.94894036925479$\\
            $5$ & $0.953689879905671$ & $0.0463101200943289$ & $0.976412058588285$\\
            $6$ & $0.977023652500333$ & $0.022976347499667$ & $0.988636008307362$\\
            $7$ & $0.988555818575133$ & $0.0114441814248673$ & $0.994419554185397$\\
            $8$ & $0.994288823371104$ & $0.00571117662889575$ & $0.997234473252792$\\
            $9$ & $0.997147129812271$ & $0.00285287018772873$ & $0.998623327220866$\\
            $10$ & $0.998574243145011$ & $0.00142575685498925$ & $0.999313175954041$\\
            $11$ & $0.999287290970944$ & $0.00071270902905618$ & $0.999656964676004$\\
            $12$ & $0.99964368781498$ & $0.00035631218502008$ & $0.999828576660796$\\
            $13$ & $0.99982185448684$ & $0.000178145513159977$ & $0.999914307580552$\\
            $14$ & $0.999910929887816$ & $0.0000890701121841753$ & $0.999957161749242$\\
            $15$ & $0.999955465603578$ & $0.0000445343964224909$ & $0.999978539377947$\\
            $16$ & $0.999977732967117$ & $0.0000222670328825503$ & $0.999989342051222$\\
            $17$ & $0.999988866519551$ & $0.0000111334804494545$ & $0.999993952227437$\\
            $18$ & $0.999994433260763$ & $0.00000556673923679529$ & $0.999995592183484$\\
            $19$ & $0.999997216609442$ & $0.00000278339055759247$ & $0.999988890662244$\\
            $20$ & $0.999998608271376$ & $0.00000139172862412273$ & $0.999976286169489$
        \end{tabular}
    \end{center}
\end{table}\\
MATLAB code to generate results:
\begin{verbatim}
    n = 20;
    xn = zeros(n, 1);
    en = ones(n, 1);
    kn = zeros(n, 1);
    f_x = @(x) exp(x) - exp(1)*x;
    f_x_p = @(x) exp(x) - exp(1);

    for i = 2:n
        xn(i) = xn(i - 1) - f_x(xn(i - 1))/f_x_p(xn(i - 1));
        en(i) = abs(xn(i) - 1);
        if i >= 3
            kn(i) = log(en(i)/en(i - 1))/log(en(i - 1)/en(i - 2));
        end
    end
    disp(xn) % Newton sequence
    disp(en) % Absolute error of Newton sequence
    disp(kn) % Approximation of rate of convergence using n, n-1 and n-2
\end{verbatim}
The results of table 3 show three key trends. Firstly, $x_n$ appears to approach 1 as $n$ increases, so 
the algorithm appears to be converging on the root $r$. Equivalently, the error $e_n$ appears to be approaching zero.
Finally, letting $k$ be the rate of convergence of the Newton sequence of interest and $d>0$, by definition we have
\begin{align*}
    \lim_{n\to +\infty}\frac{|x_{n+1}-r|}{|x_{n}-r|^k}=d.
\end{align*}
We found in problem 2.b that $k$ can be approximated by the formula
\begin{align*}
    k_n=\frac{\ln\left(\frac{|x_{n+1}-r|}{|x_n-r|}\right)}{\ln\left(\frac{|x_n-r|}{|x_{n-1}-r|}\right)}\approx k
\end{align*}
Where $\lim_{n\to +\infty}k_n=k$. In table 3, $k_n$ is very nearly 1 so we argue that $k\approx 1$. By these three
trends we have compelling reason to believe that the Newton sequence $\{x_n\}_{n\geq 1}$ is converging
to $r=1$ with a linear rate of convergence.
\subsection*{c) \normalfont Show that the rate of convergence of $\{x_n\}_{n\geq 1}$ is, in fact, linear.}
{\bf Proof.} Let $g(x)$ be a continuous function with continuous derivatives up to order two and a unique root $r$ so that $g(r)=0$. 
Suppose also that this root has multiplicity $2$. Then we know that for some at least twice differentiable 
function $h(x)$, $g(x)$ can be written as
\begin{align*}
    g(x)&=(x-r)^2h(x)\\
    \Rightarrow g'(x)&=2(x-r)h(x)+(x-r)^2h'(x)\\
    \Rightarrow g''(x)&=2h(x)+4(x-r)h'(x)+(x-r)^2h''(x)\\
    \Rightarrow g''(r)&=2h(r).
\end{align*}
Where in particular we have that $g(r)=0$, $g'(r)=0$, $g''(r)=2h(r)\neq 0$. Next, define the sequence 
$\{x_n\}_{n\geq 1}$ to be the one generated by the Newton algorithm such that
\begin{align*}
    \begin{cases}
        x_1=a\\
        x_{n+1}=x_n-\frac{g(x_n)}{g'(x_n)}, \text{\indent $n\geq 1$.} 
    \end{cases}
\end{align*}
Finally, suppose we know that in the limit the sequence converges to root $r$. We seek to show that this convergence is linear,
which by the definition of linear rate of convergence would imply that
\begin{align*}
    \lim_{n\to+\infty}\frac{|x_{n+1}-r|}{|x_n-r|}=k,\text{\indent $k>0$.}
\end{align*}
By Newton's algorithm, we have that
\begin{align*}
    x_{n+1}&=x_n-\frac{g(x_n)}{g'(x_n)}\\
    \Rightarrow x_{n+1}-r&=(x_n-r)-\frac{g(r)-g(x_n)}{g'(r)-g'(x_n)}.
\end{align*}
Next, by substituting the Taylor expansion of $g(r)$ and $g'(r)$, we derive a new expression for $x_{n+1}-r$.
\begin{align*}
    x_{n+1}-r&=(x_n-r)-\frac{(r-x_n)g'(x_n)+\frac{1}{2}(r-x_n)^2g''(\theta_1)}{(r-x_n)g''(\theta_2)}\\
    &=(x_n-r)-\left(\frac{g'(x_n)+\frac{1}{2}(r-x_n)g''(\theta_1)}{g''(\theta_2)}\right)
\end{align*}
where $\theta_1$, $\theta_2\in I[x_n,r]$. Note that $lim_{n\to+\infty}I[x_n,r]=\{r\}$, so in the limit, $\theta_1,\theta_2\in\{r\}\Rightarrow\theta_1,\theta_2=r.$
Then the desired result can be shown by
\begin{align*}
    \frac{x_{n+1}-r}{x_n-r}&=1-\left(\frac{g'(x_n)+\frac{1}{2}(r-x_n)g''(\theta_1)}{g''(\theta_2)(x_n-r)}\right)\\
    &=1+\frac{g''(\theta_1)}{2g''(\theta_2)}-\frac{g'(x_n)}{g''(\theta_2)(x_n-r)}\\
    &=1+\frac{g''(\theta_1)}{2g''(\theta_2)}-\frac{2(x_n-r)h(x_n)+(x_n-r)^2h'(x_n)}{g''(\theta_2)(x_n-r)}\\
    &=1+\frac{g''(\theta_1)}{2g''(\theta_2)}-\frac{2h(x_n)+(x_n-r)h'(x_n)}{g''(\theta_2)}\\
    \Rightarrow \lim_{n\to+\infty}\left|\frac{x_{n+1}-r}{x_n-r}\right|&=\left|1+\frac{g''(r)}{2g''(r)}-\frac{2h(r)}{g''(r)}\right|\\
    &=\left|1+\frac{1}{2}-\frac{g''(r)}{g''(r)}\right|\\
    &=\frac{1}{2}>0.
\end{align*}
So by definition, the rate of convergence of sequence $\{x_n\}_{n\geq 1}$ is linear.\\\\
Note that this theorem applies to $f(x)=e^x-ex$ from part a) since $f(x)$ has two continuous derivatives and a unique root of multiplicity 2.

\subsection*{d)\normalfont Let $\{z_n\}_{n\geq 1}$ be the sequence generated by the modified Newton's algorithm
\begin{align*}
    z_1=0,\indent z_{n+1}=z_n-p\frac{f(z_n)}{f'(z_n)},\indent n\geq 1
\end{align*}
where $p$ is the multiplicity of the root $r$ of $f(z)$. Verify that the convergence of $\{z_n\}_{n\geq 1}$ is 
quadratic.}
To verify this, we consider the original equation $f(z)=e^z-ez$, which has root $r=1$ with multiplicity $p=2$.
By precisely the same method as in 4.b, we numerically approximate the rate of convergence $k$ of $\{z_n\}_{n\geq 1}$
by the formula
\begin{align*}
    k_{n+1}=\frac{\ln\left(\frac{|z_{n+1}-r|}{|z_n-r|}\right)}{\ln\left(\frac{|z_n-r|}{|z_{n-1}-r|}\right)}\approx k.
\end{align*}
{\bf Table 4.} Approximation of rate of convergence $k_n\approx k$ of $\{z_n\}_{n\geq 1}$. MATLAB code used to generate
the results is given beneath the table.
\begin{table}[h!]
    \begin{center}
        \begin{tabular}{c|c}
            $n$ & $k_n$\\
            \hline
            $1$ & $-$\\
            $2$ & $-$\\
            $3$ & $1.9911702809902$\\
            $4$ & $1.99987570511093$\\
            \hline
            $5$ & $1.84752119194048$\\
            $6$ & $-1.21522541247082$\\
            $7$ & $-0.714058334947209$\\
            $8$ & $0$\\
            $9$ & $NaN$\\
        \end{tabular}
    \end{center}
\end{table}\\
MATLAB code used to generate results
\begin{verbatim}
    format longg
    n = 20;
    xn = zeros(n, 1);
    en = ones(n, 1);
    kn = zeros(n, 1);
    f_x = @(x) exp(x) - exp(1)*x;
    f_x_p = @(x) exp(x) - exp(1);

    for i = 2:n
        xn(i) = xn(i - 1) - 2*f_x(xn(i - 1))/f_x_p(xn(i - 1));
        en(i) = abs(xn(i) - 1);
        if i >= 3
            kn(i) = log(en(i)/en(i - 1))/log(en(i - 1)/en(i - 2));
        end
    end
    disp(xn) % Newton sequence
    disp(en) % Absolute error of Newton sequence
    disp(kn) % Approximation of rate of convergence using n, n-1 and n-2
\end{verbatim}
As was the case in 2.b, approximations $k_n$ are numerically unstable when working with limited
precision -- in this case the approximation includes division by zero after only 9 iterations. By iteration
5 the error decreases from $\sim10^{-6}$ to $\sim10^{-12}$, and remains relatively constant thereafter.
For this reason, we consider only the values above the second horizontal line which are computed before the
onset of numerical istability. Interestingly, these approximations indicate that $k\approx1.999$, suggesting 
that $\{z_n\}_{n\geq 1}$ is exhibiting quadratic convergence. 
\subsection*{e)\normalfont Show that the rate of convergence of $\{z_n\}_{n\geq 1}$ is, in fact, quadratic.}
{\bf Proof.} Let $g(z)$ be an a continuous function with continuous derivatives up to at least order three and a unique root $r$ so that $g(r)=0$ and $g'''(r)\neq 0$. 
Suppose also that root $r$ has multiplicity $2$. Then we know that for some at least thrice differentiable 
function $h(z)$, $g(z)$ can be written as
\begin{align*}
    g(z)&=(z-r)^2h(z)\\
    \Rightarrow g'(z)&=2(z-r)h(z)+(z-r)^2h'(z)\\
    \Rightarrow g''(z)&=2h(z)+4(z-r)h'(z)+(z-r)^2h''(z)\\
    \Rightarrow g'''(z)&=6h'(z)+6(z-r)h''(z)+(z-r)^2h'''(z)\\
    \Rightarrow g''(r)&=2h(r)\text{\indent and\indent}g'''(r)=6h'(r)
\end{align*}
Where in particular we have that $g(r)=0$, $g'(r)=0$, $g''(r)=2h(r)\neq 0$ and $g'''(r)=6h'(r)\neq 0$. Next, define the sequence 
$\{z_n\}_{n\geq 1}$ to be the one generated by the modified Newton algorithm such that
\begin{align*}
    \begin{cases}
        z_1=a\\
        z_{n+1}=z_n-2\frac{g(z_n)}{g'(z_n)}, \text{\indent $n\geq 1$.} 
    \end{cases}
\end{align*}
Finally, suppose we know that in the limit the sequence converges to root $r$. We seek to show that this convergence is quadratic,
which by the definition of quadratic rate of convergence would imply that
\begin{align*}
    \lim_{n\to+\infty}\frac{|z_{n+1}-r|}{|z_n-r|^2}=k,\text{\indent $k>0$.}
\end{align*}
By the modified Newton's algorithm, we have that
\begin{align*}
    z_{n+1}&=z_n-2\frac{g(z_n)}{g'(z_n)}\\
    \Rightarrow z_{n+1}-r&=(z_n-r)-2\frac{g(r)-g(z_n)}{g'(r)-g'(z_n)}.
\end{align*}
Next, by substituting the Taylor expansion of $g(r)$ and $g'(r)$, we derive a new expression for $z_{n+1}-r$.
\begin{align*}
    z_{n+1}-r&=(z_n-r)-2\frac{(r-z_n)g'(z_n)+\frac{1}{2}(r-z_n)^2g''(\theta_1)}{(r-z_n)g''(\theta_2)}\\
    &=(z_n-r)-2\left(\frac{g'(z_n)+\frac{1}{2}(r-z_n)g''(\theta_1)}{g''(\theta_2)}\right)
\end{align*}
where $\theta_1$, $\theta_2\in I[z_n,r]$. Note that $lim_{n\to+\infty}I[x_n,r]=\{r\}$, so in the limit, $\theta_1,\theta_2\in\{r\}\Rightarrow\theta_1,\theta_2=r.$
Then the desired result can be shown by
\begin{align*}
    \frac{z_{n+1}-r}{(z_n-r)^2}&=\frac{1}{z_n-r}-2\left(\frac{g'(z_n)+\frac{1}{2}(r-z_n)g''(\theta_1)}{g''(\theta_2)(z_n-r)^2}\right)\\
    &=\frac{1}{z_n-r}-2\left(\frac{2(z_n-r)h(z_n)+(z_n-r)^2h'(z_n)+\frac{1}{2}(r-z_n)g''(\theta_1)}{g''(\theta_2)(z_n-r)^2}\right)\\
    &=\frac{1}{z_n-r}-2\left(\frac{2h(z_n)+(z_n-r)h'(z_n)-\frac{1}{2}g''(\theta_1)}{g''(\theta_2)(z_n-r)}\right)\\
    &=\frac{g''(\theta_2)+g''(\theta_1)-4h(z_n)-2(z_n-r)h'(z_n)}{g''(\theta_2)(z_n-r)}\\
    \Rightarrow \lim_{n\to+\infty}\left|\frac{z_{n+1}-r}{(z_n-r)^2}\right|&=\left|\frac{2g''(r)-4h(r)-2(r-r)h'(r)}{g''(r)(r-r)}\right|\\
    &=\left|\frac{2g''(r)-2g''(r)-2(r-r)h'(r)}{g''(r)(r-r)}\right|\\
    &=\left|\frac{-2(r-r)h'(r)}{(r-r)g''(r)}\right|\\
    &=\left|\frac{-2h'(r)}{g''(r)}\right|\\
    &=2\left|\frac{h'(r)}{g''(r)}\right|=\frac{1}{3}\left|\frac{g'''(r)}{g''(r)}\right|>0,\text{\indent since $g'''(r)\neq 0$ and $g''(r)\neq 0$.}
\end{align*}
Therefore, we have that $\{z_n\}_{n\geq 1}$ converges to root $r$ of $g(z)=0$ with quadratic order of convergence.\\\\
Note that this theorem applies to $f(x)=e^x-ex$ from part a) since $f(x)$ has three continuous derivatives and a unique root $r$ of multiplicity 2 with $f'''(r)=f'''(1)=e\neq0$.

\end{document}