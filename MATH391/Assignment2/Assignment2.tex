\documentclass[11pt, letterpaper]{article}
\usepackage[margin=1.5cm]{geometry}
\pagestyle{plain}

\usepackage{amsmath, amsfonts, amssymb, amsthm}
\usepackage[shortlabels]{enumitem}
\usepackage[makeroom]{cancel}
\usepackage{graphicx}
\graphicspath{{./images/}}

\begin{document}
\title{Assignment 2\\\normalsize MATH391}
\author{Connor Braun}

\allowdisplaybreaks

\theoremstyle{definition}
\newtheorem*{prf}{Proof}
\newtheorem{recipe}{Recipe}
\newtheorem*{sol}{Solution}
\newtheorem{case}{Case}
\newtheoremstyle{mythrm}
    {0pt}{0pt}
    {\hangindent 2.5em}
    {}
    {\bfseries}
    {.}
    {0.5em}
    {}
\theoremstyle{mythrm}
\newtheorem{lemma}{Lemma}

\maketitle
\section*{Problem 1}
\subsection*{a) \normalfont Let $x_0<x_1<x_2$ and let $f(x)$ be a function that is twice differentiable in $[x_0,x_2]$.
Let $p(x)$ be a cubic polynomial so that $p(x)=a+bx+cx^2+dx^3$ for some $a,b,c,d\in\mathbb{R}$. Furthermore, suppose that $p(x_0)=f(x_0)$,
$p(x_2)=f(x_2)$, $p'(x_1)=f'(x_1)$, $p''(x_1)=f''(x_1)$.}
Then we have the following system of linear equations in $a,b,c,d$.
\begin{align*}
    p(x_0)&=a+bx_0+cx_0^2+dx_0^3=f(x_0)\\
    p(x_2)&=a+bx_2+cx_2^2+dx_2^3=f(x_2)\\
    p'(x_1)&=b+2cx_1+3dx_1^2=f'(x_1)\\
    p''(x_1)&=2c+6dx_1=f''(x_1).
\end{align*}
We can rewrite the system in matrix form as follows:
\begin{align*}
   \begin{bmatrix}
       1 & x_0 & x_0^2 & x_0^3 \\
       1 & x_2 & x_2^2 & x_2^3 \\
       0 & 1 & 2x_1 & 3x_1^2 \\
       0 & 0 & 2 & 6x_1
   \end{bmatrix} 
   \begin{bmatrix}
       a\\
       b\\
       c\\
       d
   \end{bmatrix}
   =
   \begin{bmatrix}
       f(x_0)\\
       f(x_2)\\
       f'(x_1)\\
       f''(x_1)
   \end{bmatrix}
   =\vec{f}\in\mathbb{R}^4
\end{align*}
Let $U$ be the matrix in the left hand side of this equation. In order for this system to have a unique solution $\forall\vec{f}\in\mathbb{R}^4$, it is sufficient to show
that
\begin{align*}
    det(U)=det\left(
   \begin{bmatrix}
       1 & x_0 & x_0^2 & x_0^3 \\
       1 & x_2 & x_2^2 & x_2^3 \\
       0 & 1 & 2x_1 & 3x_1^2 \\
       0 & 0 & 2 & 6x_1
   \end{bmatrix} 
    \right)\neq 0.
\end{align*}
Indeed, we have that
\begin{align*}
    det\left(
   \begin{bmatrix}
       1 & x_0 & x_0^2 & x_0^3 \\
       1 & x_2 & x_2^2 & x_2^3 \\
       0 & 1 & 2x_1 & 3x_1^2 \\
       0 & 0 & 2 & 6x_1
   \end{bmatrix} 
    \right)&=det\left(
   \begin{bmatrix}
       x_2 & x_2^2 & x_2^3\\
       1 & 2x_1 & 3x_1^2\\
       0 & 2 & 6x_1
   \end{bmatrix}
   \right)-det\left(
   \begin{bmatrix}
       x_0 & x_0^2 & x_0^3\\
       1 & 2x_1 & 3x_1^2\\
       0 & 2 & 6x_1
   \end{bmatrix}
   \right)\\
   &=x_2(6x_1^2)-(6x_1x_2^2-2x_2^3)-x_0(6x_1^2)+(6x_1x_0^2-2x_0^3)\\
   det(U)&=x_1^2(6x_2-6x_0)+x_1(6x_0^2-6x_2^2)+2(x_2^3-x_0^3)
\end{align*}
Which is a quadratic polynomial in $x_1$. Let $\Delta$ be the discriminant of this polynomial. Then,
\begin{align*}
    \Delta&=(6x_0^2-6x_2^2)^2-4(6x_2-6x_0)(2x_2^3-2x_0^3)\\
    &=(36x_0^4-72x_0^2x_2^2+36x_2^4)-4(12x_2^4-12x_0^3x_2-12x_0x_2^3+12x_0^4)\\
    &=36x_0^4-72x_0^2x_2^2+36x_2^4-48x_2^4+48x_0^3x_2+48x_0x_2^3-48x_0^4\\
    &=-12x_0^4+48x_0^3x_2-72x_0^2x_2^2+48x_0x_2^3-12x_2^4\\
    &=-12(x_0^4-4x_0^3x_2+6x_0^2x_2^2-4x_0x_2^3+x_2^4)\\
    &=-12(x_0-x_2)^4 \text{\indent (by the binomial theorem).}
\end{align*}
Where $\forall_{x_0,x_2}\in\mathbb{R}$ we have that $(x_0-x_2)^4>0$ since $x_0<x_2$. This means that $\Delta<0$, 
so $det(U)$ has no real roots and furthermore $det(U)=x_1^2(6x_2-6x_0)+x_1(6x_0^2-6x_2^2)+2(x_2^3-x_0^3)\neq0$ $\forall_{x_0,x_1,x_2}\in\mathbb{R}$, provided $x_0<x_1<x_2$
as previously supposed.
\subsection*{b) \normalfont Let $x_0<x_1<x_2$ and $f(x)$ be a function which is twice differentiable in $[x_0,x_2]$.
Formulate the third degree polynomial $\xi(x)$ that meets the following interpolatory conditions:
\begin{align*}
\xi(x_0)&=f(x_0)\\
\xi(x_2)&=f(x_2)\\
\xi'(x_1)&=f'(x_1)\\
\xi''(x_1)&=f''(x_1).
\end{align*}}
To construct $\xi(x)$ we first seek to meet the interpolatory conditions $\xi(x_0)=f(x_0)$ and $\xi(x_2)=f(x_2)$.
We begin by treating this as a Lagrange interpolation problem. Let $r(x)$ be the linear Newton form interpolating polynomial defined by
\begin{align*}
r(x)=a+b(x-x_0).
\end{align*}
For some $a,b\in\mathbb{R}$. We now impose the interpolatory conditions to find that
\begin{align*}
    r(x_0)&=a=f(x_0)\text{,\indent and}\\
    r(x_2)&=a+b(x_2-x_0)=f(x_2)\\
    \Rightarrow f(x_2)&=f(x_0)+b(x_2-x_0)\\
    \Rightarrow b&=\frac{f(x_2)-f(x_0)}{(x_2-x_0)}\\
    &=f[x_0,x_2].
\end{align*}
Where $f[x_0,x_2]$ is the divided difference formula between points $x_0$ and $x_2$. We next try to find
a third-degree polynomial $q(x)$ such that
\[\xi(x)=r(x)+q(x).\]
Now, since $r(x)$ already interpolates $f(x)$ at $x_0$ and $x_2$, we find that
\begin{align*}
    f(x_0)&=\xi(x_0)=r(x_0)+q(x_0)\\
    &=f(x_0)+q(x_0)\text{\indent (since $r(x_0)=f(x_0)$)}\\
    &\Rightarrow q(x_0)=0
\end{align*}
and similarly
\begin{align*}
    f(x_2)&=\xi(x_2)=r(x_2)+q(x_2)\\
    &=f(x_2)+q(x_2)\text{\indent (since $r(x_2)=f(x_2)$)}\\
    &\Rightarrow q(x_2)=0
\end{align*}
so $q(x)$ is a third degree polynomial with roots at $x=x_0$ and $x=x_2$. Hence, $q(x)$ takes the form
\[q(x)=(x-x_0)(x-x_2)(Ax+B)\]
for some $A,B\in\mathbb{R}$. To finish constructing $\xi(x)$, we determine $A$ and $B$ by imposing the 
last two interpolatory conditions on $\xi(x)$. First, derivatives of $\xi(x)$ are
\begin{align*}
    \xi'(x)&=f[x_0,x_2]+(x-x_2)(Ax+B)+(x-x_0)(Ax+B)+A(x-x_0)(x-x_2)\\
    \xi''(x)&=2A(x-x_2)+2A(x-x_0)+2(Ax+B)\\
    &=2A(x-x_2)+2A(x-x_0)+2Ax+2B\\
    &=2A(3x-x_0-x_2)+2B.
\end{align*}
Next, imposing the interpolatory conditions yields the following system of linear equations in $A$ and $B$
\begin{align*}
    \xi'(x_1)&=f[x_0,x_2]+(x_1-x_2)(Ax_1+B)+(x_1-x_0)(Ax_1+B)+A(x_1-x_0)(x_1-x_2)=f'(x_1)\\
    \Rightarrow f'(x_1)-f[x_0,x_2]&=(Ax_1^2+Bx_1-Ax_1x_2-Bx_2)+(Ax_1^2+Bx_1-Ax_0x_1-Bx_0)\\
    &+(Ax_1^2-Ax_1x_2-Ax_0x_1+Ax_0x_2)\\
    \Rightarrow f'(x_1)-f[x_0,x_2]&=A(3x_1^2-2x_1x_2-2x_0x_1+x_0x_2)+B(2x_1-x_2-x_0)\\\\
    \xi''(x_1)&=2A(3x_1-x_0-x_2)+2B=f''(x_1).
\end{align*}
This can be rewritten in matrix form as
\begin{align*}
    \begin{bmatrix}
        (3x_1^2-2x_1x_2-2x_0x_1+x_0x_2) & (2x_1-x_2-x_0)\\
        2(3x_1-x_0-x_2) & 2
    \end{bmatrix}
    \begin{bmatrix}
        A\\
        B
    \end{bmatrix}
    =\begin{bmatrix}
        f'(x_1)-f[x_0,x_2]\\
        f''(x_1)
    \end{bmatrix}
\end{align*}
Letting $V$ be the matrix in the left hand side of the equation, we first find $det(V)$.
\begin{align*}
    det(V)&=det\left(\begin{bmatrix}
        (3x_1^2-2x_1x_2-2x_0x_1+x_0x_2) & (2x_1-x_2-x_0)\\
        2(3x_1-x_0-x_2) & 2
    \end{bmatrix}\right)\\
    &=2(3x_1^2-2x_1x_2-2x_0x_1+x_0x_2)-2(2x_1-x_2-x_0)(3x_1-x_0-x_2).
\end{align*}
Expanding and then collecting like terms of $det(V)$ yields a quadratic polynomial in $x_0$
\begin{align*}
    det(V)=-2(x_0^2+x_0(x_2-3x_1)+(3x_1^2-3x_1x_2+x_2^2)).
\end{align*}
Next, let $\Delta$ be the discriminant of $det(V)$. We find $\Delta$ by the following
\begin{align*}
    \Delta&=(x_2-3x_1)^2-4(3x_1^2-3x_1x_2+x_2^2)\\
    &=x_2^2-6x_1x_2+9x_1^2-12x_1^2+12x_1x_2-4x_2^2\\
    &=-3x_2^2+6x_1x_2-3x_1^2\\
    &=-3(x_2^2-2x_1x_2+x_1^2)\\
    &=-3(x_2-x_1)^2\text{\indent (by the binomial theorem).}
\end{align*}
Provided $x_1<x_2$, we have that $\forall_{x_1,x_2}\in\mathbb{R}$, $(x_2-x_1)^2>0$, so $\Delta<0$. This means that 
$det(V)$ has no real roots, and thus since $x_0<x_1<x_2$, we have that $\forall_{x_0,x_1,x_2}\in\mathbb{R}$, $det(V)\neq0$. 
This allows us to compute $A$ and $B$ directly by way of Cramer's rule.
\begin{align*}
    A&=\frac{det\left(\begin{bmatrix}
    f'(x_1)-f[x_0,x_2] & (2x_1-x_2-x_0)\\
    f''(x_1) & 2\end{bmatrix}   
    \right)}{det(V)}\\
    A&=\frac{2(f'(x_1)-f[x_0,x_2])-f''(x_1)(2x_1-x_2-x_0)}{-2(x_0^2+x_0(x_2-3x_1)+(3x_1^2-3x_1x_2+x_2^2))}.\\\\
    B&=\frac{det\left(\begin{bmatrix}
    (3x_1^2-2x_1x_2-2x_0x_1+x_0x_2) & f'(x_1)-f[x_0,x_2]\\
    2(3x_1-x_0-x_2) & f''(x_1)\end{bmatrix}   
    \right)}{det(V)}\\
    B&=\frac{f''(x_1)(3x_1^2-2x_1x_2-2x_0x_1+x_0x_2)-2(f'(x_1)-f[x_0,x_2])(3x_1-x_0-x_2)}{-2(x_0^2+x_0(x_2-3x_1)+(3x_1^2-3x_1x_2+x_2^2))}.\\\\
\end{align*}
Finally, we arrive at the explicit formula for $\xi(x)$ meeting the specified interpolatory conditions on $f(x)$
\begin{align*}
    &\xi(x)=f(x_0)+f[x_0,x_2](x-x_0)+(x-x_0)(x-x_2)(\zeta)
\end{align*}
Where $\zeta$ is given by
\begin{align*}
    \zeta&=\frac{x(2(f'(x_1)-f[x_0,x_2])-f''(x_1)(2x_1-x_2-x_0))}{-2(x_0^2+x_0(x_2-3x_1)+(3x_1^2-3x_1x_2+x_2^2))}\\
    &+\frac{f''(x_1)(3x_1^2-2x_1x_2-2x_0x_1+x_0x_2)-2(f'(x_1)-f[x_0,x_2])(3x_1-x_0-x_2)}{-2(x_0^2+x_0(x_2-3x_1)+(3x_1^2-3x_1x_2+x_2^2))}.
\end{align*}
\section*{Problem 2}
\subsection*{a) \normalfont Let $f(x)$ be a function with as many derivatives as needed at and around the point $x_0$. 
Derive a difference formula approximating $f'(x_0)$ which uses the points $x_0-h$, $x_0$, $x_0+h$, $x_0+2h$ for $h\in\mathbb{R}$ small.}
We begin by finding the degree 3 Lagrange form polynomial $p_3(x)$ interpolating $f(x)$ at the points $x_0-h$, $x_0$, $x_0+h$, $x_0+2h$.
\begin{align*}
    p_3(x)&=L_{3,0}(x)f(x_0-h)+L_{3,1}(x)f(x_0)+L_{3,2}(x)f(x_0+h)+L_{3,3}(x)f(x_0+2h)\\
    &=\frac{(x-x_0)(x-x_0-h)(x-x_0-2h)}{(x_0-h-x_0)(x_0-h-x_0-h)(x_0-h-x_0-2h)}f(x_0-h)\\
    &+\frac{(x-x_0+h)(x-x_0-h)(x-x_0-2h)}{(x_0-x_0+h)(x_0-x_0-h)(x_0-x_0-2h)}f(x_0)\\
    &+\frac{(x-x_0+h)(x-x_0)(x-x_0-2h)}{(x_0+h-x_0+h)(x_0+h-x_0)(x_0+h-x_0-2h)}f(x_0+h)\\
    &+\frac{(x-x_0+h)(x-x_0)(x-x_0-h)}{(x_0+2h-x_0+h)(x_0+2h-x_0)(x_0+2h-x_0-h)}f(x_0+2h)
\end{align*}
The derivative of which is easily computable as 
\begin{align*}
    p_3'(x)&=L_{3,0}'(x)f(x_0-h)+L_{3,1}'(x)f(x_0)+L_{3,2}'(x)f(x_0+h)+L_{3,3}'(x)f(x_0+2h)\\
    &=\frac{(x-x_0-h)(x-x_0-2h)+(x-x_0)(x-x_0-2h)+(x-x_0)(x-x_0-h)}{(x_0-h-x_0)(x_0-h-x_0-h)(x_0-h-x_0-2h)}f(x_0-h)\\
    &+\frac{(x-x_0-h)(x-x_0-2h)+(x-x_0+h)(x-x_0-2h)+(x-x_0+h)(x-x_0-h)}{(x_0-x_0+h)(x_0-x_0-h)(x_0-x_0-2h)}f(x_0)\\
    &+\frac{(x-x_0)(x-x_0-2h)+(x-x_0+h)(x-x_0-2h)+(x-x_0+h)(x-x_0)}{(x_0+h-x_0+h)(x_0+h-x_0)(x_0+h-x_0-2h)}f(x_0+h)\\
    &+\frac{(x-x_0)(x-x_0-h)+(x-x_0+h)(x-x_0-h)+(x-x_0+h)(x-x_0)}{(x_0+2h-x_0+h)(x_0+2h-x_0)(x_0+2h-x_0-h)}f(x_0+2h)
\end{align*}
Where since $f(x)=p_3(x)+E(x)$, where $E(x)$ is an error term depending on $x$, we have that $f(x)\approx p_3(x)$ so
$f'(x)\approx p_3'(x)$. Furthermore, $f'(x_0)\approx p_3'(x_0)$, so
\begin{align*}
    f'(x_0)\approx p_3'(x_0)
    &=\frac{(-h)(-2h)}{(-h)(-2h)(-3h)}f(x_0-h)\\
    &+\frac{(-h)(-2h)+(h)(-2h)+(h)(-h)}{(h)(-h)(-2h)}f(x_0)\\
    &+\frac{(h)(-2h)}{(2h)(h)(-h)}f(x_0+h)\\
    &+\frac{(h)(-h)}{(3h)(2h)(h)}f(x_0+2h)\\
    &=\frac{2h^2}{-6h^3}f(x_0-h)+\frac{-h^2}{2h^3}f(x_0)+\frac{-2h^2}{-2h^3}f(x_0+h)+\frac{-h^2}{6h^3}f(x_0+2h)\\
    &=\frac{-1}{3h}f(x_0-h)+\frac{-1}{2h}f(x_0)+\frac{1}{h}f(x_0+h)+\frac{-1}{6h}f(x_0+2h)
\end{align*}
So we arrive at a formula for the approximation of $f'(x_0)$ using the four points
\begin{align*}
    f'(x_0)\approx p_3'(x)&=\frac{-2f(x_0-h)-3f(x_0)+6f(x_0+h)-f(x_0+2h)}{6h}.
\end{align*}
\subsection*{b) \normalfont Determine the order of convergence of the approximation derived in 2.a.}
We begin by constructing the Taylor expansion of $f(x_0-h)$, $f(x_0)$, $f(x_0+h)$, $f(x_0+2h)$
\begin{align*}
    &f(x_0-h)=f(x_0)-f'(x_0)h+\frac{f''(x_0)h^2}{2!}-\frac{f'''(x_0)h^3}{3!}+\frac{f^{(4)}(\theta_1)h^4}{4!}\\
    &f(x_0)=f(x_0)\\
    &f(x_0+h)=f(x_0)+f'(x_0)h+\frac{f''(x_0)h^2}{2!}+\frac{f'''(x_0)h^3}{3!}+\frac{f^{(4)}(\theta_2)h^4}{4!}\\
    &f(x_0+2h)=f(x_0)+2f'(x_0)h+2f''(x_0)h^2+\frac{4f'''(x_0)h^3}{3}+\frac{2f^{(4)}(\theta_3)h^4}{3}
\end{align*}
for some $\theta_1\in I(x_0,x_0-h)$, $\theta_2\in I(x_0,x_0+h)$, $\theta_3\in I(x_0,x_0+2h)$. Next we can scale
these by their numerator coefficients in the approximation from 2.a and add the result to find that
\begin{align*}
    -2f(x_0-h)-3f(x_0)+6f(x_0+h)-f(x_0+2h)&=(-2-3+6-1)f(x_0)\\
    &+(2+6-2)f'(x_0)h\\
    &+(-1+3-2)f''(x_0)h^2\\
    &+(\frac{1}{3}+1-\frac{4}{3})f'''(x_0)h^3\\
    &+(-\frac{f^{(4)}(\theta_1)}{12}+\frac{3f^{(4)}(\theta_2)}{12}-\frac{8f^{(4)}(\theta_3)}{12})h^4
\end{align*}
Which, letting $E=3f^{(4)}(\theta_2)-f^{(4)}(\theta_1)-8f^{(4)}(\theta_3)$, reduces to
\begin{align*}
    &-2f(x_0-h)-3f(x_0)+6f(x_0+h)-f(x_0+2h)=6f'(x_0)h+\frac{1}{12}Eh^4\\
    &\Rightarrow -2f(x_0-h)-3f(x_0)+6f(x_0+h)-f(x_0+2h)-\frac{1}{12}Eh^4=6f'(x_0)h\\
    &\Rightarrow f'(x_0)=\frac{-2f(x_0-h)-3f(x_0)+6f(x_0+h)-f(x_0+2h)}{6h}-\frac{1}{72}Eh^3
\end{align*}
Where now it is clear that the error term between $f'(x_0)$ and the approximation derived in 2.a is proportional
to $h^3$, and so the approximation has order of convergence $3$.\newpage
\section*{Problem 3}
\subsection*{a)\normalfont Let $f(x)$ be a function with as many derivatives as required at and around a point
$x_0$. Find an error term for the difference formula
\begin{align*}
f''(x_0)\approx \frac{-f(x_0+3h)+4f(x_0+2h)-5f(x_0+h)+2f(x_0)}{h^2}
\end{align*}}
We begin by writing the Taylor expansions of $-f(x_0+3h)$, $4f(x_0+2h)$ and $-5f(x_0+h)$.
\begin{align*}
    -f(x_0+3h)&=-f(x_0)-3f'(x_0)h-\frac{9f''(x_0)h^2}{2!}-\frac{3^3f'''(x_0)h^3}{3!}-\frac{3^4f^{(4)}(\theta_1)h^4}{4!}\\
    4f(x_0+2h)&=4f(x_0)+8f'(x_0)h+\frac{16f''(x_0)h^2}{2!}+\frac{32f'''(x_0)h^3}{3!}+\frac{4(2^4)f^{(4)}(\theta_2)h^4}{4!}\\
    -5f(x_0+h)&=-5f(x_0)-5f'(x_0)h-\frac{5f''(x_0)h^2}{2!}-\frac{5f'''(x_0)h^3}{3!}-\frac{5f^{(4)}(\theta_3)h^4}{4!}.
\end{align*}
For some $\theta_1\in I(x_0,x+3h)$, $\theta_2\in I(x_0,x_0+2h)$, $\theta_3\in I(x_0,x_0+h)$.
Next we add these and rearrange them to isolate $f''(x_0)$
\begin{align*}
    &f''(x_0)h^2-\frac{1}{4!}\left(64f^{(4)}(\theta_2)-81f^{(4)}(\theta_1)-5f^{(4)}(\theta_3)\right)h^4=-3f(x_0+3h)+4f(x_0+2h)-5f(x_0+h)+2f(x_0)\\
    &f''(x_0)=\frac{-3f(x_0+3h)+4f(x_0+2h)-5f(x_0+h)+2f(x_0)}{h^2}+\frac{1}{4!}\left(64f^{(4)}(\theta_2)-81f^{(4)}(\theta_1)-5f^{(4)}(\theta_3)\right)h^2\\
    &\Rightarrow f''(x_0)-\frac{-3f(x_0+3h)+4f(x_0+2h)-5f(x_0+h)+2f(x_0)}{h^2}=\frac{1}{4!}\left(64f^{(4)}(\theta_2)-81f^{(4)}(\theta_1)-5f^{(4)}(\theta_3)\right)h^2
\end{align*}
So we have that the given finite difference formula has an error term $E(h)$ given by
\begin{align*}
    E(h)=\frac{1}{4!}\left(64f^{(4)}(\theta_2)-81f^{(4)}(\theta_1)-5f^{(4)}(\theta_3)\right)h^2
\end{align*}
For some $\theta_1\in I(x_0,x+3h)$, $\theta_2\in I(x_0,x_0+2h)$, $\theta_3\in I(x_0,x_0+h)$.

\section*{Problem 4}
Let $f(t)$ be a function and suppose that $a,b\in\mathbb{R}$ and that the closed interval $[a,b]$ is subdivided into $n\in\mathbb{N}$ equal subintervals of length 
$h=\frac{b-a}{n}$. Let $Q_s^{c,n}(f)$ be the $n$-interval composite Simpson's rule on [a,b]. 
\subsection*{a) \normalfont Find $Q_s^{c,n}(f)$.}
First we divide the definite integral of $f(t)$ on $[a,b]$ into $n$ subintervals of equal size.
\begin{align*}
    \int_a^b f(t)dt=\int_a^{a+h} f(t)dt +\int_{a+h}^{a+2h}f(t)dt +\dots+ \int_{a+(n-1)h}^{a+nh}f(t)dt.
\end{align*}
This sum can be rewritten as
\begin{align*}
    \int_a^b f(t)dt=\sum_{i=1}^n \int_{a+(i-1)h}^{a+ih}f(t)dt.
\end{align*}
Where now we can use Simpson's rule to approximate each integral in the sum as follows
\begin{align*}
    \sum_{i=1}^n \int_{a+(i-1)h}^{a+ih}f(t)dt\approx \sum_{i=1}^n &\Big(\frac{a+ih-a-(i-1)h}{6}f(a+(i-1)h)\\
    &+\frac{2(a+ih-a-(i-1)h)}{3}f(\frac{2a+(2i-1)h}{2})\\
    &+\frac{a+ih-a-(i-1)h}{6}f(a+ih)\Big)
\end{align*}
\newpage
Furthermore, simplifying the right hand side we can manipulate the sum to show that
\begin{align*}
    &\sum_{i=1}^n \Big(\frac{h}{6}f(a+(i-1)h)+\frac{2h}{3}f(a+\frac{(2i-1)h}{2})+\frac{h}{6}f(a+ih)\Big)\\
    &=\frac{h}{6}\sum_{i=1}^n f(a+(i-1)h)+\frac{2h}{3}\sum_{i=1}^n f(a+\frac{(2i-1)h}{2})+\frac{h}{6}\sum_{i=1}^n f(a+ih)\\
    &=\frac{h}{6}(f(a)+f(a+h)+\dots+f(a+(n-1)h))\\
    &\indent+\frac{2h}{3}(f(a+\frac{h}{2})+f(a+\frac{3h}{2})+\dots+f(a+\frac{(2n-1)h}{2}))\\
    &\indent+\frac{h}{6}(f(a+h)+f(a+2h)+\dots+f(a+nh))\\
    &=\frac{h}{6}(f(a)+f(a+nh))\\
    &\indent+\frac{h}{6}(f(a+h)+f(a+2h)+\dots+f(a+(n-1)h))\\
    &\indent+\frac{2h}{3}(f(a+\frac{h}{2})+f(a+\frac{3h}{2})+\dots+f(a+\frac{(2n-1)h}{2}))\\
    &\indent+\frac{h}{6}(f(a+h)+f(a+2h)+\dots+f(a+(n-1)h))\\
    &=\frac{h}{6}(f(a)+f(a+nh))\\
    &\indent+\frac{h}{3}(f(a+h)+f(a+2h)+\dots+f(a+(n-1)h))\\
    &\indent+\frac{2h}{3}(f(a+\frac{h}{2})+f(a+\frac{3h}{2})+\dots+f(a+\frac{(2n-1)h}{2})).
\end{align*}
Finally, since $h=\frac{b-a}{n}$, we have that $a+nh=b$, we arrive at the solution
\begin{align*}
    Q_s^{c,n}(f)&=\frac{h}{6}\big(f(a)+f(b)\big)\\
    &+\frac{h}{3}\big(f(a+h)+f(a+2h)+\dots+f(a+(n-1)h)\big)\\
    &+\frac{2h}{3}\big(f(a+\frac{h}{2})+f(a+\frac{3h}{2})+\dots+f(a+\frac{(2n-1)h}{2})\big).
\end{align*}
\subsection*{b) \normalfont Suppose that $f(t)$ has a continuous fourth derivative and let $\theta\in[a,b]$. Show that}
\begin{align*}
    \int_a^b f(t)dt=Q_S^{c,n}(f)-\frac{(b-a)^5}{2880n^4}f^{(4)}(\theta).
\end{align*}
As before, we begin by dividing the definite integral of $f(t)$ on interval $[a,b]$ into $n$ subintervals of equal size.
\begin{align*}
    \int_a^b f(t)dt=\int_a^{a+h} f(t)dt +\int_{a+h}^{a+2h}f(t)dt +\dots+ \int_{a+(n-1)h}^{a+nh}f(t)dt.
\end{align*}
Let $Q_S(g,m,k)$ be the Simpson's quadrature formula for some function $g$ on interval $[m,k]$. Then, by the error of the Simpson's quadrature (theorem), we have that
\begin{align*}
    \int_a^b f(t)dt&=Q_S(f,a,a+h)-\frac{f^{(4)}(\theta_1)h^5}{2880}\\
    &+Q_S(f,a+h,a+2h)-\frac{f^{(4)}(\theta_2)h^5}{2880}+\dots\\
    &\dots+Q_S(f,a+(n-1)h,a+nh)-\frac{f^{(4)}(\theta_n)h^5}{2880}
\end{align*}
For some $\theta_1\in[a,a+h]$, $\theta_2\in[a+h,a+2h]$,...,$\theta_n\in[a+(m-1)h,a+nh]$. This can be rewritten as
\begin{align*}
    \int_a^bf(t)dt&=Q_S^{c,n}(f)-\frac{h^5}{2880}(f^{(4)}(\theta_1)+f^{(4)}(\theta_2)+\dots+f^{(4)}(\theta_n))\\
    &=Q_S^{c,n}(f)-\frac{nh^5}{2880}\left(\frac{1}{n}\sum_{j=1}^n f^{(4)}(\theta_j)\right)
\end{align*}
Where now since $f^{(4)}(\theta_j)$ is continuous for all $1\leq j\leq n$, the intermediate value theorem guarantees $\exists_\theta\in I[\theta_1,\theta_2,\dots,\theta_n]$ such that
\begin{align*}
    \int_a^bf(t)dt&=Q_S^{c,n}(f)-\frac{nh^5}{2880}f^{(4)}(\theta).
\end{align*}
However, since $h=\frac{b-a}{n}$, we can rewrite this formula as
\begin{align*}
    \int_a^bf(t)dt=Q_S^{c,n}(f)-\frac{(b-a)^5}{2880n^4}f^{(4)}(\theta).
\end{align*}
So we have derived the desired equality.
\subsection*{c)\normalfont Suppose that $-9\leq f^{(4)}(t)\leq 2$ $\forall_t\in[0,2]$. Find the smallest number of 
subintervals $n$ required to guarantee that}
\[\left|\int_0^2f(t)dt-Q_S^{c,n}(f)\right|\leq10^{-5}.\]
First we apply the result from 4.b
\begin{align*}
    \left|\int_0^2f(t)dt-Q_S^{c,n}(f)\right|=\left|\frac{2^5}{2880n^4}f^{(4)}(\theta)\right|
\end{align*}
for some $\theta\in[0,2]$. Applying the constraint on the error
\begin{align*}
    \left|\frac{2^5}{2880n^4}f^{(4)}(\theta)\right|\leq 10^{-5}
\end{align*}
Now, since $n^4>0$ and $\forall_{\theta}$, $-9\leq f^{(4)}(\theta)\leq 2$, we can solve for $n$ directly
\begin{align*}
    &\left|\frac{2^5}{2880n^4}f^{(4)}(\theta)\right|\leq 10^{-5}\\
    &\Rightarrow 9\left|\frac{32}{2880}\right|\leq n^410^{-5}\\
    &\Rightarrow \frac{10^5}{10}\leq n^{4}\\
    &\Rightarrow \sqrt[4]{10000}\leq n\\
    &\Rightarrow n\geq 10
\end{align*}
Therefore, to bound the error between $\int_0^2 f(t)dt$ and $Q_S^{c,n}(f)$ by $10^{-5}$, we require that the interval $[0,2]$ be subdivided
into at least $n=10$ equal subintervals. 
\newpage
\section*{Problem 5}
Function for evaluating definite integrals using composite Simpson's rule. Function specifications are as described in 
the assignment.
\begin{verbatim}
    function Q = Composite_Simpson(fname, a, b, n)

        h = (b - a)/n;
        Q = 0; 
        for i = 1:n
            sub_int_a = a + (i-1)*h; 
            sub_int_b = a + i*h; 
            sub_int_mid = (sub_int_a + sub_int_b)/2; 
            
            Q = Q + (h/6)*fname(sub_int_a) + ...
                (2*h/3)*fname(sub_int_mid) + ...
                (h/6)*fname(sub_int_b); 
        end
    end
\end{verbatim} 
\end{document}























% \section*{Problem 1}
% \subsection*{a) \normalfont Let $x_0<x_1<x_2$ and let $f(x)$ be a function that is twice differentiable in $[x_0,x_2]$.
% Let $p(x)$ be a cubic polynomial so that $p(x)=a+bx+cx^2+dx^3$ for some $a,b,c,d\in\mathbb{R}$. Furthermore, suppose that $p(x_0)=f(x_0)$,
% $p(x_2)=f(x_2)$, $p'(x_1)=f'(x_1)$, $p''(x_1)=f''(x_1)$.}
% Then we have the following system of linear equations in $a,b,c,d$.
% \begin{align*}
%     p(x_0)&=a+bx_0+cx_0^2+dx_0^3=f(x_0)\\
%     p(x_2)&=a+bx_2+cx_2^2+dx_2^3=f(x_2)\\
%     p'(x_1)&=b+2cx_1+3dx_1^2=f'(x_1)\\
%     p''(x_1)&=2c+6dx_1=f''(x_1).
% \end{align*}
% We can rewrite the system in matrix form as follows:
% \begin{align*}
%    \begin{bmatrix}
%        1 & x_0 & x_0^2 & x_0^3 \\
%        1 & x_2 & x_2^2 & x_2^3 \\
%        0 & 1 & 2x_1 & 3x_1^2 \\
%        0 & 0 & 2 & 6x_1
%    \end{bmatrix} 
%    \begin{bmatrix}
%        a\\
%        b\\
%        c\\
%        d
%    \end{bmatrix}
%    =
%    \begin{bmatrix}
%        f(x_0)\\
%        f(x_2)\\
%        f'(x_1)\\
%        f''(x_1)
%    \end{bmatrix}
%    =\vec{f}\in\mathbb{R}^4
% \end{align*}
% In order for this system to have a unique solution $\forall\vec{f}\in\mathbb{R}^4$, it is sufficient to show
% that
% \begin{align*}
%     det\left(
%    \begin{bmatrix}
%        1 & x_0 & x_0^2 & x_0^3 \\
%        1 & x_2 & x_2^2 & x_2^3 \\
%        0 & 1 & 2x_1 & 3x_1^2 \\
%        0 & 0 & 2 & 6x_1
%    \end{bmatrix} 
%     \right)\neq 0.
% \end{align*}
% Indeed, we have that
% \begin{align*}
%     det\left(
%    \begin{bmatrix}
%        1 & x_0 & x_0^2 & x_0^3 \\
%        1 & x_2 & x_2^2 & x_2^3 \\
%        0 & 1 & 2x_1 & 3x_1^2 \\
%        0 & 0 & 2 & 6x_1
%    \end{bmatrix} 
%     \right)&=det\left(
%    \begin{bmatrix}
%        x_2 & x_2^2 & x_2^3\\
%        1 & 2x_1 & 3x_1^2\\
%        0 & 2 & 6x_1
%    \end{bmatrix}
%    \right)-det\left(
%    \begin{bmatrix}
%        x_0 & x_0^2 & x_0^3\\
%        1 & 2x_1 & 3x_1^2\\
%        0 & 2 & 6x_1
%    \end{bmatrix}
%    \right)\\
%    &=x_2(6x_1^2)-(6x_1x_2^2-2x_2^3)-x_0(6x_1^2)+(6x_1x_0^2-2x_0^3)\\
%    &=6x_1^2(x_2-x_0)+6x_1(x_0^2-x_2^2)+2(x_2^3-x_0^3)
% \end{align*}
% which is never $0$ since we have that $x_0<x_1<x_2$.

% \subsection*{b) \normalfont Find a formula for $p(x)$ from 1.a.}
% By Cramer's rule we can determine a closed formula for coefficient $d$ in $p(x)$ as follows
% \begin{align*}
%     d =\frac{
%     det\left(
%     \begin{bmatrix}
%        1 & x_0 & x_0^2 & f(x_0) \\
%        1 & x_2 & x_2^2 & f(x_2) \\
%        0 & 1 & 2x_1 & f'(x_1) \\
%        0 & 0 & 2 & f''(x_1)
%     \end{bmatrix}
%     \right)}{
%     det\left(
%     \begin{bmatrix}
%        1 & x_0 & x_0^2 & x_0^3 \\
%        1 & x_2 & x_2^2 & x_2^3 \\
%        0 & 1 & 2x_1 & 3x_1^2 \\
%        0 & 0 & 2 & 6x_1
%     \end{bmatrix}        
%     \right)
%     }
% \end{align*}
% The denominator was determined in 1.a, so we seek to compute the numerator.
% \begin{align*}
%     &det\left(
%     \begin{bmatrix}
%        1 & x_0 & x_0^2 & f(x_0) \\
%        1 & x_2 & x_2^2 & f(x_2) \\
%        0 & 1 & 2x_1 & f'(x_1) \\
%        0 & 0 & 2 & f''(x_1)
%     \end{bmatrix}
%     \right)
%     =det\left(
%         \begin{matrix}
%             x_2 & x_2^2 & f(x_2) \\
%             1 & 2x_1 & f'(x_1) \\
%             0 & 2 & f''(x_1)
%         \end{matrix}
%     \right)-det\left(\begin{bmatrix}
%         x_0 & x_0^2 & f(x_0) \\
%         1 & 2x_1 & f'(x_1) \\
%         0 & 2 & f''(x_1) 
%     \end{bmatrix}
%     \right)\\
%     &=x_2(2x_1f''(x_1)-2f'(x_1))-x_2^2f''(x_1)+2f(x_2)-x_0(2x_1f''(x_1)-2f'(x_1))+(x_0^2f''(x_1)-2f(x_0))\\
%     &=2x_1x_2f''(x_1)-2x_2f'(x_1)-x_2^2f''(x_1)+2f(x_2)-2x_0x_1f''(x_1)+2x_0f'(x_1)+x_0^2f''(x_1)-2f(x_0)\\
%     &=f''(x_1)(2x_1x_2+x_0^2-x_2^2-2x_0x_1)+f'(x_1)(2x_0-2x_2)+2(f(x_2)-f(x_0))
% \end{align*}
% Therefore, by Cramer's rule we have that
% \begin{align*}
%     d=\frac{f''(x_1)(2x_1x_2+x_0^2-x_2^2-2x_0x_1)+f'(x_1)(2x_0-2x_2)+2(f(x_2)-f(x_0))}{6x_1^2(x_2-x_0)+6x_1(x_0^2-x_2^2)+2(x_2^3-x_0^3)}
% \end{align*}
% Having determined coefficient $d$, we can proceed to define coefficients $c$, $b$ and $a$ by backward substitution. Reconsidering
% the original system
% \begin{align*}
%    \begin{bmatrix}
%        1 & x_0 & x_0^2 & x_0^3 \\
%        1 & x_2 & x_2^2 & x_2^3 \\
%        0 & 1 & 2x_1 & 3x_1^2 \\
%        0 & 0 & 2 & 6x_1
%    \end{bmatrix} 
%    \begin{bmatrix}
%        a\\
%        b\\
%        c\\
%        d
%    \end{bmatrix}
%    =
%    \begin{bmatrix}
%        f(x_0)\\
%        f(x_2)\\
%        f'(x_1)\\
%        f''(x_1)
%    \end{bmatrix}
%    =\vec{f}\in\mathbb{R}^4
% \end{align*}
% We can see that
% \begin{align*}
%     2c+6x_1d&=f''(x_1)\\
%     \Rightarrow c&=\frac{1}{2}(f''(x_1)-6x_1d)\\
%     b+2x_1c+3x_1^2d&=f'(x_1)\\
%     \Rightarrow b&=f'(x_1)-2x_1c-3x_1^2d\\
%     a+x_2b+x_2^2c+x_2^3d&=f(x_2)\\
%     \Rightarrow a&=f(x_2)-x_2b-x_2^2c-x_2^3d
% \end{align*}
% Where each coefficient has been defined in terms of those previously determined. Substituting in the explicit formulas would produce
% a very unweildy equation for $p(x)$, so instead we'll keep it symbolic and define the formula for $p(x)$ as follows.
% \begin{align*}
%     &p(x)=a+bx+cx^2+dx^3\\
%     &\text{where}\\
%     &d=\frac{f''(x_1)(2x_1x_2+x_0^2-x_2^2-2x_0x_1)+f'(x_1)(2x_0-2x_2)+2(f(x_2)-f(x_0))}{6x_1^2(x_2-x_0)+6x_1(x_0^2-x_2^2)+2(x_2^3-x_0^3)}\\
%     &c=\frac{1}{2}(f''(x_1)-6x_1d)\\
%     &b=f'(x_1)-2x_1c-3x_1^2d\\
%     &a=f(x_2)-x_2b-x_2^2c-x_2^3d\\
%     &\text{so we can define a formula for $p(x)$ as}\\
%     &p(x)=f(x_2)-x_2b-x_2^2c-x_2^3d+x(f'(x_1)-2x_1c-3x_1^2d)+x^2(\frac{1}{2}(f''(x_1)-6x_1d))+x^3(d).
% \end{align*}