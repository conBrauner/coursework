\documentclass[11pt, letterpaper]{article}
\usepackage[margin=1.5cm]{geometry}
\pagestyle{plain}

\usepackage{amsmath, amsfonts, amssymb, amsthm}
\usepackage[shortlabels]{enumitem}
\usepackage[makeroom]{cancel}
\usepackage{graphicx}
\graphicspath{{./images/}}

\begin{document}
\title{Assignment 2\\\normalsize STAT321}
\author{Connor Braun}

\allowdisplaybreaks

\maketitle

\section*{Problem 1}
\subsection*{a) \normalfont Compute $E[X]$. Let $x_1=-10$, $x_2=0$ and $x_3=5$. Then let $1\leq i\leq 3$ where $i\in \mathbb{Z}$. Then}
\[E[X]=\sum_{i=1}^{3}x_iP(X=x_i)=-10(\frac{3}{10})+0(\frac{1}{2})+5(\frac{1}{5})=-3+1=-2\]
Solution: $E[X]=-2$.
\subsection*{b) \normalfont Compute $VAR[X]$.}
\[VAR[X]=E[X^2]-E[X]^2=100(\frac{3}{10})+0(\frac{1}{2})+25(\frac{1}{5})-(-2)^2=31\]
Solution: $VAR[X]=31$.
\subsection*{c) \normalfont Compute $E[3X]$}
\[E[3X]=3E[X]=3(-2)=-6\]
Solution: $E[3X]=-6$.

\section*{Problem 2}
\subsection*{a) \normalfont Construct a probability distribution table for $X$.}
$X\sim binomial(4,\space \frac{1}{2})$. Therefore, we have that
\begin{align*}
    P(X=0)&={4\choose 0}(\frac{1}{2})^0(1-\frac{1}{2})^4=\frac{1}{16}\\
    P(X=1)&={4\choose 1}(\frac{1}{2})^1(1-\frac{1}{2})^3=\frac{1}{4}\\
    P(X=2)&={4\choose 2}(\frac{1}{2})^2(1-\frac{1}{2})^2=\frac{3}{8}\\
    P(X=3)&={4\choose 3}(\frac{1}{2})^3(1-\frac{1}{2})^1=\frac{1}{4}\\
    P(X=4)&={4\choose 4}(\frac{1}{2})^4(1-\frac{1}{2})^0=\frac{1}{16}
\end{align*}
So a probability distribution table for the described experiment could be 
\begin{table}[h!]
    \begin{center}
        \begin{tabular}{c|c|c|c|c|c}
            $x$ & 0 & 1 & 2 & 3 & 4\\
            \hline
            $P(X=x)$ & $\frac{1}{16}$ & $\frac{1}{4}$ & $\frac{3}{8}$ & $\frac{1}{4}$ & $\frac{1}{16}$
        \end{tabular}
    \end{center}
\end{table}
\subsection*{b) \normalfont Create a probability distribution graph for $X$.}
\begin{center}
  \makebox[\textwidth]{\includegraphics[width=150mm]{figure_1.png}}
\end{center}
The probability distribution graph of random variable $X$ is symmetric.
\subsection*{c) \normalfont Compute $P(X\geq 2)$.}
\[P(X\geq 2)=P(X=2)+P(X=3)+P(X=4)=\frac{3}{8}+\frac{1}{4}+\frac{1}{16}=\frac{11}{16}\]
Solution: $P(X\geq 2)=\frac{11}{16}$.

\section*{Problem 3}
Let $X$ be the random variable for how many defective screws are found in a box. 
We have that $X\sim binomial(20,\space \frac{1}{100})$.
\subsection*{a) \normalfont Compute $P(X=0)$.}
\[P(X=0)={20\choose 0}\left(\frac{1}{100}\right)^0\left(1-\frac{1}{100}\right)^{20}=(99/100)^20=0.81791\]
Solution: $P(X=0)=0.81791$.
\subsection*{b) \normalfont Compute $P(X=1)$.}
\[P(X=1)={20\choose 1}\left(\frac{1}{100}\right)\left(\frac{99}{100}\right)^{19}=0.16523\]
Solution $P(X=1)=0.16523$.
\subsection*{c) \normalfont Compute $P(X>1)$.}
$P(X>1)=1-P(X>1)^C=1-P(X\leq 1)=1-(P(X=0)+P(X=1))$.
\begin{align*}
P(X>1)&=1-0.81791-0.16523 \text{\indent (from part 3.a and 3.b).}\\
&=0.01686
\end{align*}
Solution: the probability that any given box of screws will be eligible for a refund is $0.0169$, or $1.69\%$.

\section*{Problem 4}
Let $H_1$, $H_2$ be the events that the first and second coin (respectively) lands heads up. Then let
$P(T_1)=1-P(H_1)=P(H_1)^C$ and $P(T_2)=1-P(H_2)=P(H_2)^C$ be the probability that the first and second 
coin (respectively) land tails. Then $P(H_1)=0.7$, $P(T_1)=0.3$, $P(H_2)=0.8$ and $P(T_2)=0.2$. Furthermore,
\begin{align*}
    P(X=2)&=P(H_1\cap H_2)=(0.7)(0.8)=0.56\\
    P(X=1)&=P(H_1\cap T_2) + P(T_1\cap H_2)=(0.7)(0.2)+(0.3)(0.8)=0.38\\
    P(X=0)&=P(T_1\cap T_2)=(0.3)(0.2)=0.06
\end{align*}
\subsection*{a) \normalfont Compute $E[X]$.}
\begin{align*}
E[X]&=2P(X=2)+P(X=1)+0P(X=0)\\
&=2(0.56)+1(0.38)+0\\
&=1.5
\end{align*}
Solution: $E[X]=1.5$.
\subsection*{b) \normalfont Compute $SD[X]$.}
\begin{align*}
    SD[X]&=\sqrt{VAR[X]}\\
    &=\sqrt{E[X^2]-E[X]^2}\\
    &=\sqrt{4(0.56)+1(0.38)-\frac{9}{4}}\\
    &=0.60828
\end{align*}
Solution: $SD[X]=0.60828$.

\section*{Problem 5}
Let $X$ be the number of trials for Bob to win $r$ rounds. Suppose that the probability of
winning is $\frac{1}{6}$.
\subsection*{a) \normalfont Let $r=1$. Compute $P(X=4)$.}
Here we have that $X\sim neg.\space binomial(1,\space \frac{1}{6})$, so 
\[P(X=4)={3\choose 0}\left(\frac{1}{6}\right)\left(\frac{5}{6}\right)^3=0.09645\]
Solution: The probability that Bob's first win is on the fourth round is $0.09645$.
\subsection*{b) \normalfont Let $r=2$. Compute $P(X=3)$.}
Now we have that $X\sim neg.\space binomial(2,\space \frac{1}{6})$, so
\[P(X=3)={2\choose 1}\left(\frac{1}{6}\right)^2\left(\frac{5}{6}\right)^1=0.04629\]
Solution: The probability that Bob's second win is on the third round is $0.04629$.

\section*{Problem 6}
Note that $X\sim binomial(2,\space \frac{2}{9})$.
\subsection*{a) \normalfont Create a probability distribution table for $X$.}
\begin{align*}
    P(X=0)&={2\choose 0}\left(\frac{2}{9}\right)^0\left(\frac{7}{9}\right)^2=0.60494\\
    P(X=1)&={2\choose 1}\left(\frac{2}{9}\right)^1\left(\frac{7}{9}\right)^1=0.34568\\
    P(X=2)&={2\choose 2}\left(\frac{2}{9}\right)^2\left(\frac{7}{9}\right)^0=0.04938
\end{align*}
\begin{table}[h!]
    \begin{center}
        \begin{tabular}{c|c|c|c}
            $x$ & 0 & 1 & 2\\
            \hline
            $P(X=x)$ & $0.60494$ & $0.34568$ & $0.04938$ 
        \end{tabular}
    \end{center}
\end{table}
\subsection*{b) \normalfont Compute $E[X]$.}
For any discrete random variable $Y$ where $Y\sim binomial(n,\space p)$, we have that $E[Y]=np$. Hence
\[E[X]=2\left(\frac{2}{9}\right)=\frac{4}{9}.\]
Solution: $E[X]=\frac{4}{9}$.

\section*{Problem 7}
\subsection*{a) Compute $E[4Y-2]$.}
\begin{align*}
    E[4Y-2]&=E[4Y]+E[-2]\\
    &=4E[Y]-2\\
    &=4\left(120\times \frac{3}{10}\right)-2\\
    &=142
\end{align*}
Solution: $E[4Y-2]=142$.
\subsection*{b) Compute $VAR[4Y-2]$.}
\begin{align*}
    VAR[4Y-2]&=VAR[4Y]+VAR[-2]\\
    &=16VAR[Y]\\
    &=16\left(120\times\left(\frac{3}{10}\right)\left(\frac{7}{10}\right)\right)\\
    &=403.2
\end{align*}
Solution: $VAR[X]=403.2$.

\section*{Problem 8}
Let $Y$ be the random variable for the number of yellow M\&M's in a pack of 23, $p=0.24=\frac{6}{25}$. Then $Y\sim binomial(23,\space \frac{6}{25})$.
\subsection*{a) \normalfont Compute $P(Y>10)$.}
\[P(Y>10)=\sum_{n=11}^{23}P(Y=n)=\sum_{n=11}^{23}{23\choose n}p^n(1-p)^{23-n}\]
We will compute this with the following code in $R$:
\begin{verbatim}
    x<-11:23
    p<-sum(dbinom(x, 23, 0.24))
\end{verbatim}
Where here the variable $p=P(Y>10)$ takes the value $0.01087513$.\\
Solution: $P(Y>10)=0.01087513$.
\subsection*{b) \normalfont Compute $P(5\leq Y\leq 10)$.}
\[P(5\leq Y\leq 10)=\sum_{n=5}^{10}P(Y=n)=\sum_{n=5}^{10}{23\choose n}p^n(1-p)^{23-n}\]
We will compute this with the following code in $R$:
\begin{verbatim}
    x<-5:10
    p<-sum(dbinom(x, 23, 0.24))
\end{verbatim}
Where here the variable $p=P(5\leq Y\leq 10)$ takes the value $0.6674023$.\\
Solution: $P(5\leq Y\leq 10)=0.6674023$.
\subsection*{c) \normalfont Compute $P(Y=7|5\leq Y\leq 10)$.}
By the law of total probability, we have that
\begin{align*}
    P(Y=7)&=P(Y=7\cap (5\leq Y\leq 10))+P(Y=7\cap(5\leq Y\leq 10)^C)\\
\end{align*}
However, the events that $Y=7$ and $(5\leq Y\leq 10)^C$ are mutually exclusive, since $Y$ cannot simultaneously
be $7$ and not in the interval $[5,10]$. Hence, $P(Y=7\cap 5\leq Y\leq 10)=0$, so we have
\[P(Y=7)=P(Y=y\cap 5\leq Y\leq 10)\]
Then we can compute $P(Y=7|5\leq Y\leq 10)$ directly:
\begin{align*}
    P(Y=7|5\leq Y\leq 10)&=\frac{P(Y=7\cap 5\leq Y\leq 10)}{P(5\leq Y\leq 10)}\\
    &=\frac{P(Y=7)}{P(5\leq Y\leq 10)}\\
    &=\frac{{23\choose 7}p^7(1-p)^{16}}{\sum_{n=5}^{10}{23\choose n}p^n(1-p)^{23-n}}\\
    &=\frac{0.1393}{0.6674023} \text{\indent (taking the denominator from 8.b).}\\
    &=0.20872
\end{align*}
Solution: $P(Y=7|5\leq Y\leq 10)=0.20872$.

\section*{Problem 9}
\subsection*{a) \normalfont Let $Y$ be the number of hit targets out of 3 trials. The probability that Bob 
hits the target is 0.7, so $Y\sim binomial(3,\space 0.7)$. Compute $P(Y=3)$.}
\[P(Y=3)={3\choose 3}(0.7)^3(0.3)^0=0.343\]
Solution: $P(Y=3)=0.343$.
\subsection*{b) \normalfont Let $X=Y+1$ be the random variable for the number of the trial on which
Bob misses the target. Compute $E[Y]$.}
First, we have that $X\sim geometric(\frac{3}{10})$. Then, we can show that:
\begin{align*}
    P(X\geq 4)&=1-P(X\geq 4)^C\\
    &=1-P(X<4)\\
    &=1-P(X=1)-P(X=2)-P(X=3)\\
    &=1-\left(\left(\frac{7}{10}\right)^0\left(\frac{3}{10}\right)^1+\left(\frac{7}{10}\right)^1\left(\frac{3}{10}\right)^1+\left(\frac{7}{10}\right)^2\left(\frac{3}{10}\right)^1\right)\\
    &=1-\left(\frac{3}{10}\right)-\left(\frac{21}{100}\right)-\left(\frac{147}{1000}\right)\\
    &=0.3430
\end{align*}
\newpage
Then, we have that
\begin{align*}
    P(X=3)=\left(\frac{147}{1000}\right)\text{\indent (as found above in 9.b).}\\
    P(X=2)=\left(\frac{21}{100}\right)\text{\indent (as found above in 9.b).}\\
    P(X=1)=\left(\frac{3}{10}\right)\text{\indent (as found above in 9.b).}
\end{align*}
Note that $P(X\geq 4)$ is the event that Bob's first miss would occur after his third hit, i.e., after
he had already hit all three shots. Then we can construct the following probability distribution table.
\begin{table}[h!]
    \begin{center}
        \begin{tabular}{c|c|c|c|c}
            $y$ & 0 & 1 & 2 & 3\\
            \hline
            $x$ & 1 & 2 & 3 & $\geq4$\\
            \hline
            $P(X=x)$ & $\frac{3}{10}$ & $\frac{21}{100}$ & $\frac{147}{1000}$ & $0.3430$ 
        \end{tabular}
    \end{center}
\end{table}
And proceed to compute $E[Y]$ by the conventional formula:
\[E[Y]=0P(Y=0)+1P(Y=1)+2P(Y=2)+3P(Y=3)=\frac{21}{100}+\frac{294}{1000}+1.0290=1.5330\]
Solution: We'd expect Bob to hit an average of $1.5330$ shots in the long run. 
\end{document}