\documentclass[12pt, letterpaper]{article}
\usepackage[margin=2cm]{geometry}
\pagestyle{plain}

\usepackage{amsmath, amsfonts, amssymb, amsthm}
\usepackage[shortlabels]{enumitem}
\usepackage{mathptmx}
\usepackage[makeroom]{cancel}
\usepackage{indentfirst}
\usepackage{graphicx}


\newenvironment{collapsable}{}{}
\allowdisplaybreaks
\graphicspath{{~./figures/}}

\begin{document}
\begin{center}
    {\section*{\normalfont\normalsize\bf HBI Seminar Summary}}
    {\normalfont\normalsize January 28, 2022}
\end{center}

This seminar presented evidence that the  suprmammillary nucleus (SuM) conveys a signal indicating future organism locomotion to brain regions involved in memory and navigation. Furthermore, this signal appears to be upstream of other previously postulated sources and is capable of coordinating activity in disparate brain regions involved in memory and navigation. The observation that medial septum (which receives input from the SuM) activity can be strongly correlated with organism locomotion and speed motivated exploration into the role of SuM in locomotion and hippocampal activity.\\

A large subpopulation of SuM cells were found to correlate strongly to organism speed and future speed. A smaller subpopulation were found to be tightly locked to hippocampal theta, despite being spatially separated from it. Interestingly, optogenetic stimulation of the SuM reliably initiated animal locomotion, and inhibition in an already locomoting animal stopped the behavior. While hippocampal theta could be entrained to rhythmic stimulation of the SuM, total inhibition of the SuM did not impact theta oscillations, suggesting that it is not a spontaneous controller of hippocampal theta. In fact, stimulating SuM caused hippocampal cells to lose coherence with hippocampal theta, suggesting that SuM control over hippocampus is auxillary to whatever is spontaneously regulating theta oscillations.\\ 

SuM outputs can be classified as Tac1+ or Tac1- which appeared to be functionally distinct. For instance, Tac1+ cell stimulation reliably initiated locomotion, inhibited hippocampal cells that negatively correlated with speed, and excited hippocampal cells which positively correlated with speed. Conversely, Tac1- cells had no such effect. Instead, rhythmic stimulation of Tac1- cells was found to be sufficient to entrain hippocampal theta. While the functional significance of these effects remains unknown, the clear functional delineation of SuM subpopulations is fascinating. These findings appear to be of great relevance to the future of understanding how the brain internally represents navigation.
\end{document}