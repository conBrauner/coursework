\documentclass[11pt, letterpaper]{article}
\usepackage[margin=2cm]{geometry}
\pagestyle{plain}

\usepackage{amsmath, amsfonts, amssymb, amsthm}
\usepackage[shortlabels]{enumitem}
\usepackage{mathptmx}
\usepackage[makeroom]{cancel}

\allowdisplaybreaks

\title{The quantification of navigation independent neuronal phase precession}
\author{Connor Braun}
\date{\vspace{-5ex}}


\begin{document}

\maketitle
\fontsize{12}{12}\selectfont
\pagenumbering{gobble}

% Topic sentence 
{\bf Introduction.} For the last thirty years, hippocampal place cells have been consistently observed to exhibit phase 
precession with respect to theta-range (8-12 Hz) local field potential oscillations. 
% Expository sentence -- flip subject order
That is, as an organism traverses a place field, a corresponding place cell bursts progressively earlier
on each subsequent theta cycle. 
% Generality of phase precession
In accordance with the temporal coding hypothesis, phase precession has been implicated as a putative 
mechanism for encoding sequence structure of events into memory, suggesting that it could be a more 
general neurocomputational phenomenon and not strictly limited to spatial behaviors. To date, phase precession has been observed primarily during such spatial behaviors as traversing linear tracks or 
open fields, but there is very limited evidence to suggest that pyramidal cells of the CA1 exhibit phase precession 
during specific non-spatial behaviors including REM sleep and stationary head wheel running. As such, 
it is unknown to what extent phase precession dynamics might be a more general coding strategy independent of navigation.
% Research question
%The project proposed herein seeks to contribute to the determination of phase precession as a general 
%neurocomputational mechanism by quantifying it in hippocampal pyramidal neuron time series independent
%of spatial behavior.
\\

% Hypothesis
{\bf Hypothesis.} If hippocampal theta phase precession is a general mechanism for temporal encoding, then it will not only
be identifiable, but quantifiable independent of navigational behaviors.\\

{\bf Objectives.} The problem is that conventional methods of identifying phase precession rely on {\it a priori} knowledge of neural tuning so that spike timing
with respect to local theta oscillations can be fit to some function of behavior.
This curtails our ability to investigate the generality of phase precession, since neuron-behavior tuning
is typically not knowable beforehand. For this reason, we first seek to develop a behavior-agnostic algorithm 
for quantification of hippocampal theta phase precession. Next, we will construct a computational model of neuronal phase precession 
which will allow us to simulate data {\it in silico} independent of any notion of concurrent behavior. We can then proceed to characterize 
the biologically-relevant conditions under which our metric returns type I or type II error. Finally, our quantification will be used
to seek phase precession dynamics in hippocampal single-unit recordings from animals undergoing REM sleep or other
non-navigational behaviors. \\

{\bf Methodology.} The quantification algorithm will rely on theoretical underpinnings from discrete dynamical systems.
Our initial neuron modeling strategy will be to numerically integrate a stochastic leaky integrate-and-fire model with a subthreshold spike frequency adaptation term using the 
Advanced Research Computing cluster at the University of Calgary.
This model will be made to precess in phase with respect to an idealized (sinusoidal) theta rhythm by a dual linear oscillator input. Furthermore, the complete model will have parameters controlling biologically relevant dynamic phenomena, such as noise characteristics, excitability, 
spike frequency adaptation and phase precession characteristics. This will allow us to define biologically relevant conditions
under which our quantification algorithm is type I or type II erroneous.  
The complete quantification algorithm will then serve as a dependent variable to seek and quantify phase precession
in single-unit recordings (alongside concurrent extracellular local field potential recordings) taken from neurons in the hippocampal formation. 
Organism behavior at the time of these recordings will be the independent variable; categorically either navigational or non-navigational. These data will be 
obtained from the Collaborative Research in Computational Neuroscience database. Single-unit place cell recordings from entorhinal cortex, CA1 and CA3
taken from animals undergoing REM sleep will also be contributed by David Dupret lab, University of Oxford --- a close collaborator of the Nicola lab.
\\

{\bf Significance.} We expect to successfully develop a behavior independent quantification and identify navigation independent phase precession in the hippocampus, liberating it as a 
primarily navigational phenomenon. In doing so, this project will potentially yield valuable insight not only into how living systems form memories at the 
level of neural dynamics, but also into principles of neural coding at large. 

%This involves the development of a computational neuron model
%which adequately captures the behavior of place cells under an experimentally set phase relationship with
%a simulated theta rhythm. Moreover, the parameters of this model must map to biologically relevant variables, 
%such that we can define the conditions which result in type I or type II error. After the algorithm has been
%determined suitably robust to biological variaiton, it will serve as a dependent variable with which to quantify
%phase precession in navigation-independent hippocampal pyramidal time series.
\end{document}

