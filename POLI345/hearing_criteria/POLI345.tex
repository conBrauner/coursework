\documentclass[12pt, letterpaper]{article}
\usepackage[margin=2cm]{geometry}
\pagestyle{plain}

\usepackage{amsmath, amsfonts, amssymb, amsthm}
\usepackage[shortlabels]{enumitem}
\usepackage{mathptmx}
\usepackage[makeroom]{cancel}
\usepackage{enumitem}


\allowdisplaybreaks

\title{The quantification of navigation independent neuronal phase precession}
\vspace{-8ex}
\author{}
\vspace{-8ex}
\date{}


\begin{document}

\begin{center}
  {\section*{\normalfont\normalsize\bf Criteria for Pipeline evaluation}}
\end{center}

\vspace{12pt}

\noindent Commissioner assessment criteria have been divided into three weighted
categories. At the outset of each presentation, please indicate whether you are
ultimately for or against the pipeline irrespective of the terms on which you
are so. During your party’s allotted ten minutes, you are encouraged to cover as
many points as possible. Your group can still be awarded full credit in a
category for covering just one point under a heading, but only if the argument
made is exceptional. All groups MUST include point 1.a to be conferred
consideration for any category. All other points are optional, with 1.b being
worth an undisclosed bonus.

\vspace{12pt}

\section*{\normalfont\normalsize\bf 1. Conflict resolution (a:PASS/FAIL, b:
  BONUS)}
\begin{enumerate}[label={\bf\alph*}.]
\item All stakeholders must briefly state anticipated sources of future conflict
  should the commission rule in opposition to their desired outcome.
  Additionally, suggest at least one way by which your party would like to see
  this potential conflict resolved should it become a reality. This component is
  MANDATORY for the commission to lend any consideration to your party’s
  position.
\item Credit will be awarded to groups who actively identify points of
  contention during proceedings and attempt to resolve them by way of amicable
  compromise during question periods at the hearing. Alternatively, the bonus
  will be awarded to groups who collaborate with other stakeholders prior to the
  hearing to find commonalities and present them. All members of such an
  alliance must indicate which points of their presentation are shared and with
  which groups.
\end{enumerate}
\section*{\normalfont\normalsize\bf 2. Relationships \& The
  Land/Water/Environment (75\%)}
Stakeholders had ought to:
\begin{enumerate}[label={\bf\alph*}.]
  \item Incorporate opposing party's perspectives and ways of knowing into
    proposed solutions , with additional consideration awarded for solutions which
    promote the continued resurgence of Indigenous culture(s) over time.
  \item Fortifying their case by taking lessons from the failings of historic
    treaties and ongoing treaty land entitlement disputes.
  \item Present potential land/water/environmental impacts of greatest concern
    to their groups related to their position regarding the proposed pipeline.
  \item Those in favour of the pipeline should include prevention and/or
    mitigation strategies for said potential land/water/environmental impacts as
    well as address long-term and short-term environmental sustainability.
  \end{enumerate}
  \section*{\normalfont\normalsize\bf 3. Economy (25\%)}
  Stakeholder goals should be supported by:
  \begin{enumerate}[label={\bf\alph*}.]
  \item Strategy based in traditional Indigenous economics
  \item Strategy based in mainstream economics
  \item Arguments highlighting benefits to stakeholders or the broader Canadian economics


\end{document}
