\documentclass[11pt, letterpaper]{article}
\usepackage[margin=1.5cm]{geometry}
\pagestyle{plain}

\usepackage{amsmath, amsfonts, amssymb, amsthm}
\usepackage[shortlabels]{enumitem}

\begin{document}
\title{Assignment 1\\\normalsize MATH271}
\author{Connor Braun}

\theoremstyle{definition}
\newtheorem*{prf}{Proof}
\newtheorem*{sol}{Solution}
\newtheorem{case}{Case}
\newtheoremstyle{mythrm}
    {0pt}{0pt}
    {\hangindent 2.5em}
    {}
    {\bfseries}
    {.}
    {0.5em}
    {}
\theoremstyle{mythrm}
\newtheorem{lemma}{Lemma}

\maketitle
\section*{Problemset 1}

\subsection*{a) For all integers $x$, there is an integer $y$ so that $3|x+y$}
{\large\it |The statement is True.|}
\begin{prf}
    Suppose $x$ is an integer. Then, by the quotient-remainder theorem, we have that
    \[x=3q+r\]
    where $3\neq 0$ and $q$ and $r$ are integers with $r$ is on the interval $[0,3)$. Subtracting $r$ from 
    both sides yields
    \[x-r=3q\text{.}\]
    Next, we set $-r=y$. Clearly, $y$ is an integer since $r$ is an integer. With this 
    substitution the equation becomes
    \[x+y=3q\]
    which, by the definition of divisibility, implies that $3|x+y$ since $x+y,\,3,\, 
    q$ are integers. Therefore, for any integer $x$, there exists an integer $y$ such
    that $3|x+y$.
\end{prf}

\subsection*{b) For all integers $x$, there is an integer $y$ so that $3|x+y$ and $3|x-y$}
{\large\it |The statement is False.|}\\[0.25cm]
{\bf Negation:} There exists an integer $x$ such that for any integer $y$, either $3\nmid x+y$ or $3\nmid x-y$.
\begin{prf}[by contradiction]
    Consider $x=1$ where 1 is an integer and suppose $y$ is some integer. For the purposes of deriving a contradiction, suppose that both 
    $3|x+y$ and $3|x-y$. From the definition of divisibility, $3|x+y$ implies that
    \[x+y=y+1=3k\]
    for some integer $k$. Subtracting 1 from both sides reveals that
    \[y=3k-1\text{.}\]
    Similarly, by the definition of divisibility, $3|x-y$ implies that
    \[x-y=1-y=3t\]
    for some integer $t$. This further implies that
    \[y=1-3t\text{.}\]
    Now, since both $3k-1$ and $1-3t$ are equal to $y$, we can set them to be equal to 
    one another to find that
    \[3k-1=1-3t\text{.}\]
    Adding $3t+1$ to both sides yields
    \[3k+3t=2\text{.}\]
    Finally, factoring a 3 from the left hand side and dividing both sides by 3 yields
    \[k+t=\frac{2}{3}\]
    which shows that $k+t$ cannot be an integer since $\frac{2}{3}$ is not an integer.
    Since the integers are closed under addition, this equality would require that either $k$ or $t$ not be integers, violating the established
    supposition that $k$ and $t$ be integers in order for both $3|x+y$ and $3|x-y$. Hence, either 
    $3\nmid x+y$ or $3\nmid x-y$. Furthermore, since the definition of divisibility by 3 is not met for at least one of $x+y$ or 
    $x-y$, we conclude the negation of the original statement to be true by contradiction. 
\end{prf}

\subsection*{c) For all integers $x$ and $y$, if $3|x+y$ then $3|x$ or $3|y$}
{\large\it |The statement is False.|}\\[0.25cm]
{\bf Negation:} There exists two integers $x$ and $y$ such that $3|x+y$ but $3\nmid x$ and $3\nmid y$.
\begin{prf}
    Consider $x=1$ and $y=2$, where both $x$ and $y$ are integers.
    Clearly, it is the case that $3|x+y$, since $x+y$ can be written in the form
    \[x+y=1+2=3(1)\]
    where $3$ and $1$ on the right hand side are both integers with $3\neq 0$. Despite this, $x=1$ is not divisible
    by 3, since, by the quotient-remainder theorem, the unique integers 0 and 1 (where 1
    is on the interval $[0,3)$) allow us to write 1 in the form
    \[1=3(0)+1\]
    Since the quotient-remainder theorem guarantees uniqueness of integers 0 and 1, there is no way
    to write 1 in the form $1=3(k)$ with $k$ being an integer. Therefore, $3\nmid 1$ which by substitution implies that
    $3\nmid x$.\\[0.25cm]
    Similarly, 2 is not divisible by 3, since, by the quotient-remainder theorem, the unique integers 0 
    and 2 (where 2 is on the interval $[0,3)$) allow us to write 2 in the form
    \[2=3(0)+2\text{.}\]
    Just as before, since the quotient-remainder theorem guarantees uniqueness of integers 0 and 2, there is no way
    to write 2 in the form $2=3(t)$ with $t$ being an integer. Therefore, $3\nmid 2$ which by substitution implies that
    $3\nmid y$.\newline
    Moreover, in the case of $x=1$ and $y=2$, we have shown that $3|x+y$ but $3\nmid x$ and $3\nmid y$. 
\end{prf}

\subsection*{d) For all integers $x$ and $y$, if $3|xy$ then $3|x$ or $3|y$.}
{\large\it |The statement is True.|}
\begin{prf}[by contradiction]
    Suppose that $x$ and $y$ are integers and that $3|xy$. Suppose also for the purpose of deriving a contradiction 
    that $3\nmid x$ and $3\nmid y$. By the definition of divisibility, $3|xy$ implies that
    \[xy=3k\]
    for some integer k, where $3\neq 0$. By the quotient-remainder theorem, 
    $3\nmid x$ implies that 
    \[x=3q_{1}+r_{1}\]
    where $q_{1}$ and $r_{1}$ are integers, and $r_{1}$ is on the interval $[1, 3)$. 
    As a brief aside, normally the quotient-remainder theorem would specify that integer $r_{1}$ be on the interval $[0, 3)$.
    However, $r_{1}=0$ would imply that $3|x$, since then we'd have that $x=3q_{1}$ where $q_{1}$ is an integer. 
    Since we supposed that $3\nmid x$, we need to restrict $r_{1}$ to interval $[1, 3)$. 
    Similarly, by the quotient-remainder theorem, $3\nmid y$ implies that
    \[y=3q_{2}+r_{2}\] 
    where $q_{2}$ and $r_{2}$ are integers, and $r_{2}$ is on the interval $[1, 3)$ instead of $[0, 3)$ to satisfy our supposition that $3\nmid y$ 
    (by the same logic as argued for $r_{1}$ being on the interval $[1, 3)$). 
    Substituting these definitions for $x$ and $y$ into $xy=3k$, we see that
    \[(3q_{1}+r_{1})(3q_{2}+r_{2})=3k\]
    Expanding the left hand side yields
    \[9q_{1}q_{2}+3q_{1}r_{2}+3q_{2}r_{1}+r_{1}r_{2}=3k\]
    Factoring out a 3 from the left hand side and then dividing both sides by 3 yields
    \[3q_{1}q_{2}+q_{1}r_{2}+q_{2}r_{1}+\frac{1}{3}r_{1}r_{2}=k\]
    In order for this equality to hold, $3q_{1}q_{2}+q_{1}r_{2}+q_{2}r_{1}+\frac{1}{3}r_{1}r_{2}$ 
    must be an integer since $k$ is an integer. Clearly, $3q_{1}q_{2}+q_{1}r_{2}+q_{2}r_{1}$ 
    is an integer since the integers are closed under both multiplication and addition, but it is not 
    necessarily the case that $\frac{1}{3}r_{1}r_{2}$ is since $\frac{1}{3}$ is not an integer.\\[0.5cm]
    In order for $\frac{1}{3}r_{1}r_{2}$ to be an integer, $3|r_{1}r_{2}$ since then $r_{1}r_{2}=3t$
    for some integer $t$. When this is the case, $\frac{1}{3}r_{1}r_{2}$ can be rewritten as 
    \[\frac{3t}{3}=t\in \mathbb{Z}\]
    There are 4 possible cases, considering the fact that both $r_{1}\in [1,3)$ and 
    $r_{2}\in [1,3)$:
    \begin{enumerate}[\bf\text{Case} 1:]
        \item $r_{1}=1$ and $r_{2}=1$\\
            Then $r_{1}r_{2}=1$. By the quotient-remainder theorem, for dividend 1 and divisor $3\neq 0$, 
            the unique integers $q=0$ and $r=1\in[0, 3)$ satisfy the equation
            $$1=3q+r=3(0)+1$$
            Since $r\neq 0$ and these integers are unique for  dividend 1 and divisor
            3, there is no way to write $1=3(h)$ with $h$ being an integer. Therefore, when $r_{1}=1$ and $r_{2}=1$, $3\nmid r_{1}r_{2}$.
        \item $r_{1}=2$ and $r_{2}=1$\\
            Then $r_{1}r_{2}=2$. By the quotient-remainder theorem, for dividend 2 and divisor $3\neq 0$, 
            the unique integers $q=0$ and $r=2\in[0, 3)$ satisfy the equation
            $$2=3q+r=3(0)+2$$
            Since $r\neq 0$ and these integers are unique for  dividend 2 and divisor
            3, there is no way to write $2=3(h)$ with $h$ being an integer. Therefore, when $r_{1}=2$ and $r_{2}=1$, $3\nmid r_{1}r_{2}$.
        \item $r_{1}=1$ and $r_{2}=2$\\
            Then $r_{1}r_{2}=2$. This product, and hence the result, is identical to that of case 2. Therefore, when $r_{1}=1$ and $r_{2}=2$, 
            $3\nmid r_{1}r_{2}$.
        \item $r_{1}=2$ and $r_{2}=2$\\
            Then $r_{1}r_{2}=4$. By the quotient-remainder theorem, for dividend 4 and divisor $3\neq 0$, 
            the unique integers $q=1$ and $r=1\in[0, 3)$ satisfy the equation
            $$4=3q+r=3(1)+1$$
            Since $r\neq 0$ and these integers are unique for  dividend 4 and divisor
            3, there is no way to write $4=3(h)$ with $h$ being an integer. Therefore, when $r_{1}=2$ and $r_{2}=2$, $3\nmid r_{1}r_{2}$.      
    \end{enumerate}
    No matter the case, $3\nmid r_{1}r_{2}$ and hence there is no integer $t$ for which $r_{1}r_{2}=3t$. Furthermore, it is impossible for
    $\frac{1}{3}r_{1}r_{2}$ to be an integer, which means it is impossible for $3q_{1}q_{2}+q_{1}r_{2}+q_{2}r_{1}+\frac{1}{3}r_{1}r_{2}$ to be an integer. 
    Therefore, we have that 
    \[3q_{1}q_{2}+q_{1}r_{2}+q_{2}r_{1}+\frac{1}{3}r_{1}r_{2}\neq k\]
    which, working backwards, implies that
    \[(3q_{1}+r_{1})(3q_{2}+r_{2})\neq 3k\]
    which, working backwards once more, gives rise to the contradiction
    \[xy\neq 3k\]
    Which implies that $3\nmid xy$. This contradicts with our supposition that $3|xy$. Therefore, for all integers $x$ and $y$, if $3|xy$ then
    either $3|x$ or $3|y$. 
\end{prf}

\section*{Problemset 2}

\subsection*{a) If $x$ is an irrational number, then for all integers $m,n$ where $n\neq 0$, $m+nx$ is irrational}
{\large\it |The statement is True.|}
\begin{prf}[by contradiction]
    Suppose that $x$ is an irrational number and that $m,n$ are integers with $n\neq 0$. For the purpose of deriving a contradiction,
     suppose also that $m+nx$ is a rational number.By the definition of rationality, $m+nx$ can be represented by coprime integers $a,b$ where $b\neq 0$ such that
    \[m+nx=\frac{a}{b}\text{.}\]
    Multiplying both sides of the equation by $b$ yields
    \[bm+bnx=a\text{.}\]
    Next, subtracting $bm$ from both sides yields
    \[bnx=a-bm\]
    Since $b,n,a,m$ are all integers and $b\neq 0$ and $n\neq 0$, both $a-bm$ and $bn$ are integers, with $bn\neq 0$.
    To show that this equality does not hold for irrational number $x$, consider the following Lemma.\\[0.5cm]
    \begin{lemma}
        For any irrational number $y$ and any integer $h\neq 0$, $hy$ is irrational.\\
        {\bf Proof} (by contradiction). Suppose $y$ is an irrational number and $h$ is an integer where $h\neq 0$.
        For the purpose of deriving a contradiction, suppose also that the product $hy$ is rational. Then by the definition of 
        rationality, $hy$ can be represented by coprime integers $c,d$ where $d\neq 0$ such that
        \[hy=\frac{c}{d}\text{.}\]
        Dividing both sides by $h$ yields
        \[y=\frac{c}{hd}\]
        However, since it was specified that $c,h,d \in \mathbb{Z}$ and both $h\neq 0$ and $d\neq 0$, both the numerator and denominator of $\frac{c}{hd}$ are integers. Furthermore,
        $hd\neq 0$ since neither $h$ nor $d$ are 0. Hence, $\frac{c}{hd}$ meets the definition of a rational number, while $y$ was supposed to be irrational and we have that 
        \[y\neq \frac{c}{hd}\]
        and
        \[hy\neq \frac{c}{d}\]
        which contradicts the supposition that $hy$ was rational. Therefore, for any irrational number $y$, 
        no matter what integer $h\neq 0$ one selects, the product $hy$ is irrational.\\[0.5cm]
    \end{lemma}
    \noindent Lemma 1 shows that $bnx$ must be irrational since $bn\neq 0$ and is an integer while $x$ is irrational. However, it was already shown that
    $a-bm$ is an integer. This results in the inequality
    \[bnx\neq a-bm\]
    Which, working backwards, reveals that 
    \[bm+bnx\neq a\]
    and
    \[m+nx\neq \frac{a}{b}\]
    Which contradicts the supposition that $m+nx$ is rational. Therefore, $m+nx$ must be irrational for an irrational number $x$ and integers $m,n$ with $n\neq 0$.
\end{prf}

\subsection*{b) For all real numbers $x$, there is a real number $y$ so that $x+y$ is irrational.}
{\large\it |The statement is True.|}
\begin{prf}
    Suppose $x$ is a real number and consider the real number $y=\sqrt{2}-x$. Then the sum of $x$ and $y$ yields
    \[x+y=x+(\sqrt{2}-x)=\sqrt{2}\]
    Where $\sqrt{2}$ is irrational. Therefore, for any real number $x$, there exists a real number $y$ so that their sum
    $x+y$ is irrational.
\end{prf}

\subsection*{c) For all real numbers $x$ and $y$, if $x+y$ is rational then $x$ or $y$ is rational.}
{\large\it |The statement is False.|}\\[0.25cm]
{\bf Negation:} There exists real numbers $x$ and $y$ such that $x+y$ is rational and both $x$ and $y$ are irrational.
\begin{prf}
    Consider the example of $x=\sqrt{2}$ and $y=-\sqrt{2}$. By Lemma 1, $y=-\sqrt{2}$ is
    irrational since $-1$ is an integer, $-1\neq 0$, and $\sqrt{2}$ is an irrational number. This also means that $x=\sqrt{2}$ is 
    an irrational number, and both $y$ and $x$ are real numbers. Taking the sum of $x$ and $y$ yields
    \[x+y=\sqrt{2}-\sqrt{2}=0\]
    where 0 is a rational number since it can be written in the form
    \[0=\frac{0}{a}\]
    For any integer $a\neq 0$. Therefore, it is not the case that the sum of any two real numbers $x$ and $y$ being rational
    guarantees that either $x$ or $y$ be rational.
\end{prf}

\subsection*{d) For all real numbers $x$ and $y$, if $xy$ is irrational then $x$ or $y$ is irrational.}
{\large\it |The statement is True.|}
\begin{prf}[by contradiction]
    Suppose that $x$ and $y$ are both real numbers and that their product $xy$ is irrational. For the purpose of deriving a contradiction,
    suppose that $x$ and $y$ are both rational. This means that there exists integers $a,b,c,d$ with both $b\neq 0$ and $d\neq 0$.
    Then, by the definition of a rational number, we have
    \[x=\frac{a}{b}\]
    and 
    \[y=\frac{c}{d}\] 
    Taking the product $xy$ reveals that
    \[xy=\frac{ac}{bd}\] 
    The products $ac$ and $bd$ are both integers, and $bd\neq 0$ since neither $b=0$ nor $d=0$. This means that the fraction
    $\frac{ac}{bd}$ meets the definition of rationality, which contradicts the supposition that $xy$ is irrational. Thus, the supposition
    that $x$ and $y$ are both rational cannot be the case, and furthermore that for any real numbers $x$ and $y$, if their product
    is irrational, then either $x$ or $y$ must be irrational. 
\end{prf}

\section*{Problemset 3}

\subsection*{a) Use the Euclidean Algorithm to compute $\gcd{(2021, 271)}$ and use that to find integers $x$ and $y$ so that $\gcd{(2021, 271)}=2021x+271y$.}
{\large\it |Part one: compute $\gcd{(2021, 271)}$|}
\begin{sol}[by Euclidean Algorithm]
    The quotient-remainder theorem guarantees that for any integer dividend $a$ and integer divisor $d> 0$, there exits two unique integers $q$ and $r$ 
    (called the quotient and remainder respectively) with $r$ on the interval $[0,d)$ such that
    \[a=dq+r\text{.}\]
    Which allows one to make use of the fact that if $a,d,q,r\in \mathbb{Z}$ with not both $a\neq 0$ and $d\neq 0$ (satisfied by the
    requirements of the quotient-remainder theorem so long as $a\neq 0$) then we have that
    \[\gcd{(a,d)}=\gcd{(d,r)}\text{.}\]
    Now, since 2021 and 271 are integers with neither equal to 0, 2021 can be written in the form
    \[2021=7(271)+124\text{, \indent such that $\gcd{(2021,271)}=\gcd{(271,124)}$.}\]
    Reapplying the same fact that 271 and 124 are integers with neither equal to 0, 271 can be written in the form 
    \[271=2(124)+23\text{, \indent such that $\gcd{(271,124)}=\gcd{(124,23)}$.}\]
    We reapply the exact same step repeatedly until one of integers $s,t$ in $\gcd{(s, t)}$ is 0. 
    \[124=5(23)+9\text{, \indent such that $\gcd{(124,23)}=\gcd{(23,9)}$.}\]
    \[23=2(9)+5\text{, \indent such that $\gcd{(23,9)}=\gcd{(9,5)}$.}\]
    \[9=1(5)+4\text{, \indent such that $\gcd{(9,5)}=\gcd{(5,4)}$.}\]
    \[5=1(4)+1\text{, \indent such that $\gcd{(5,4)}=\gcd{(4,1)}$.}\]
    \[4=4(1)+0\text{, \indent such that $\gcd{(4,1)}=\gcd{(1,0)}$.}\]
    From which we conclude that $\gcd{(1,0)}=\gcd{(2021,271)}=1$.
\end{sol}
\noindent{\large\it |Part two: use previous result to compute integers $x$ and $y$ such that $\gcd{(2021,271)}=2021x+271y$|}
\begin{sol}
    Using the so-called table method, we solve equation $2021n+271m=r_{i}$, where $n,m,r_{i}$ are integers and $r_{i}$ is the remainder
    in each equation from part one. To start, we solve for the trivial case of 2021 and 271
    \[\text{\bf Row 1: }2021=(1)2021+(0)271\]
    and
    \[\text{\bf Row 2: }271=(0)2021+(1)271\text{.}\]
    Next we solve for 124 using equation $2021=7(271)+124$ from part one
    \[\text{\bf Row 3: }124 = (1)2021+(-7)271\text{\bf\indent (Row 1 - 7(Row 2))}\]
    Which is equivalent to a row operation which scales the linear equation for 271 by $-7$ before adding it to the equation for 2021.
    We repeat this process until we have found integers $x,y$ which satisfy equation $\gcd{(2021,271)}=2021x+271y=1$:
    \[\text{\bf Row 4: }23=(-2)2021+(15)271\text{\bf\indent (Row 2 - 2(Row 3))}\]
    \[\text{\bf Row 5: }9=(11)2021+(-82)271\text{\bf\indent (Row 3 - 5(Row 4))}\]
    \[\text{\bf Row 6: }5=(-24)2021+(179)271\text{\bf\indent (Row 4 - 2(Row 5))}\]
    \[\text{\bf Row 7: }4=(35)2021+(-261)271\text{\bf\indent (Row 5 - 1(Row 6))}\]
    \[\text{\bf Row 8: }1=(-59)2021+(440)271\text{\bf\indent (Row 6 - 1(Row 7))}\]
    Where Row 8 reveals that integers $x=-59$ and $y=440$ satisfy equation $\gcd{(2021,271)}=1=2021x+271y$.
\end{sol}

\subsection*{b) Is it true that for all positive integers $a$ and $b$, $\gcd{(a,b)}\leq \gcd{(a+b,a-b)}$?}
{\large\it |The statement is True.|}
\begin{prf}
    Suppose that $a$ and $b$ are positive integers. Suppose also that $\gcd{(a,b)}=g_{1}$ which is an integer. Since this means that $g_{1}$ divides both $a$ and $b$, we have that
    \[a=kg_{1}\text{ and }b=tg_{1}\] 
    For some integers $k$ and $t$. By the definition of greatest common factor, we have the fact that for any integer $x$ which divides both $a$ and $b$
    \[\gcd{(a,b)}\geq x\text{.}\]
    So, if both $g_{1}|a+b$ and $g_{1}|a-b$, then we will have that $\gcd{(a,b)}\leq \gcd{(a+b,a-b)}$. By substitution, we have that
    \[a+b=kg_{1}+tg_{1}\text{.}\]
    Factoring out $g_{1}$ from the right hand side yields
    \[a+b=g_{1}(k+t)\]
    Where $k+t$ is an integer. Therefore, $g_{1}|a+b$. Similarly, substituting in for $a-b$ yields
    \[a-b=g_{1}k-g_{1}t\text{.}\]
    Factoring out $g_{1}$ from the right hand side yields
    \[a-b=g_{1}(k-t)\] 
    where $k-t$ is an integer. Therefore, we also have that $g_{1}|a-b$. Furthermore, since $g_{1}=\gcd{(a,b)}$ divides both
    $a+b$ and $a-b$, it must be the case that $\gcd{(a,b)}\leq \gcd{(a+b,a-b)}$.
\end{prf}

\subsection*{c) Is it true that for all positive integers $a$ and $b$, $\gcd{(a+b,a-b)}\leq \gcd{(a,b)}$?}
{\large\it |The statement is False.|}\\[0.25cm]
{\bf Negation:} There exists positive integers $a$ and $b$ such that $\gcd{(a+b,a-b)}>\gcd{(a,b)}.$
\begin{prf}
    Consider integers $a=3$ and $b=1$. Now, using the Euclidean algorithm and quotient-remainder theorem as described in Problemset 3-c, we find that
    \[a=3=3(1)+0\text{, \indent such that $\gcd{(3,1)}=\gcd{(1,0)}$.}\]
    From which it is shown that $\gcd{(3,1)}=\gcd{(1,0)}=1$. Similarly, to determine  $\gcd{(a+b,a-b)}$, we begin with
    \[a+b=k(a-b)+r\]
    for some integers $k,r$. Substituting in $a=3$ and $b=1$ we get 
    \[4=2(2)+0\text{, \indent such that $\gcd{(4,2)}=\gcd{(2,0)}$.}\]
    From which it is shown that $\gcd{(a+b,a-b)}=\gcd{(4,2)}=\gcd{(2,0)}=2$. Hence we have that
    \[\gcd{(a+b,a-b)}>\gcd{(a,b)}\text{.}\]
    Therefore, for positive integers $a=3$ and $b=1$ we have that $\gcd{(a+b,a-b)}>\gcd{(a,b)}$.
\end{prf}
\end{document}