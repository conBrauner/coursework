\documentclass[12pt, letterpaper]{article}
\usepackage[margin=2cm]{geometry}
\pagestyle{plain}

\usepackage{amsmath, amsfonts, amssymb, amsthm}
\usepackage[shortlabels]{enumitem}
\usepackage{mathptmx}
\usepackage[makeroom]{cancel}
\usepackage{indentfirst}
\usepackage{graphicx}


\newenvironment{collapsable}{}{}
\allowdisplaybreaks
\graphicspath{{~./figures/}}

\begin{document}
\begin{center}
    {\section*{\normalfont\normalsize\bf USRA application draft}}
\end{center}
\noindent{\bf Abstract (1800 characters)}\\
\noindent {\it Provide a short abstract or overview of the proposed research project, including the objectives of the project.
Please use language for a broad audience}\\

Time flows in one direction, and we take for granted that this flow is exquisitely represented in the memories formed by our brains. Sequences of events in any form — lyrics of a song, events in a movie and so forth — are automatically preserved and recalled effortlessly, yet the neural mechanisms by which the brain encodes time are unknown. Phase precession is a phenomenon which occurs in brain structures critical to memory formation and is promising as a mechanism for bestowing memories with a temporal structure. The problem is that no tools currently exist to detect phase precession outside of animal navigational studies which inhibits our ability to study it as a general mechanism of encoding time. I have spent the last eight months developing an algorithm which not only detects, but quantifies phase precession under any behavioral condition which I call the Phase Relationship Quantification (PRQ) algorithm. However, the PRQ is cumbersome for large datasets. To enable the targeted application of this test, I propose to next develop a neural network using a supervised learning paradigm which detects phase precession over very large datasets recorded in behaving animals. First, I will label brain data for phase precession using a semi-automated approach and my previous work on phase precession detection. Next, I will design and train a neural network to detect phase precession. Finally, I will analyze non-navigational neural data with the network, compare its output to that of the PRQ and assess the usefulness of the PRQ in training similar networks for other applications. In this way, I aim to liberate phase precession as a purely navigational phenomena and create tools to aid the future of understanding how time is encoded in the brain.\\

\noindent{\bf Originality, creativity and significance (1800 characters)}\\
\noindent {\it Highlight the originality and creative aspects of your research project.
Please use language for a broad audience}\\

It has been nearly thirty years since phase precession was first discovered in navigating mice, and almost just as long since it was identified as being capable of encoding sequences in living networks of neurons. Phase precession detection remains an important task for those studying learning and memory, and is increasingly relevant in understanding the role of sleep as a mechanism of consolidating newly formed memories. Despite this, the conventional way of quantifying phase precession is with linear regression analysis which is both inefficient and confined to navigational behaviors. However, I see no reason to believe phase precession is purely navigational, and furthermore expect phase precession to look similar between any two neurons exhibiting it. Hence, it should be feasible to construct a neural network which is trained to detect phase precession in one subgroup of neurons using traditional techniques and have it reliably detect phase precession in other populations where traditional techniques would fail. To my knowledge, this is a first-of-its-kind approach, despite the coexistence of the requisite machine learning techniques and phase precession research for several decades. Additionally, phase precession could be a remarkably general phenomenon. One biologically plausible model suggests that it requires only the interaction of two electrical oscillations to emerge and could work to force groups of neurons to fire in sequence with one another. Electrical oscillations are nigh ubiquitous in brain tissue, and sequential firing is an essential ingredient for a myriad of brain functions. Hence, the value of this project is high as an untried precursor to understanding the computational significance of neural activity in many brain regions.\\

\noindent{\bf Potential benefit (1300 characters)}\\
\noindent {\it Briefly explain the potential benefit this project will provide to you}\\

Over the course of my academic journey, I’ve discovered that my passion is for artifical intelligence and computational neuroscience. This motivated me to take on mathematics as a second major in addition to neuroscience, which I intended as a way to build a strong background in the theory of applied mathematics. Hopefully, this will enable me to be flexible, creative and efficient when solving problems in my research and career as a whole. While I love my degree, it offers practically no focused training in machine learning, which I need to understand and be able to implement in a practical setting to succeed in my career aspirations. The lab in which I’d be working on this project makes frequent use of cutting edge artificial intelligence techniques and represents an invaluable opportunity for me to acquire practical experience and mentorship not available to me otherwise. Furthermore, this is likely my last summer of undergraduate research before I move on to either an internship as an artifical intelligence engineer or a graduate degree program in artifical intelligence research. An undergraduate research award would hence afford me the opportunity to develop essential practical skills to compliment my formal training and enable me to pursue my passion as a career.\\

\noindent{\bf Relevant experience (1300 characters)}\\
\noindent {\it Tell us about any relevant professional, volunteer or academic experience that prepared you for succeeding with the proposed project}\\

This project is centered on deep learning and artificial intelligence development, and hence relies on skill in both programming and mathematics. For the past three years, I’ve been programming in Python for research, coursework and recreationally for applications such as data analysis, computational modeling and some novice work in machine learning. I consider myself to be proficient with Python, and am confident I have the skills to execute this project. I also have several years of formal training in math, and excel in it as demonstrated by my transcript. I’m certainly passionate about math, but also have an aptitude for it which will be of great importance as I familiarize myself with the requisite knowledge for this project. I also have four years of training in neuroscience and about eight months of research on the topic which this project is based on. I am highly familiar with the literature on phase precession. In fact, this project is only possible given the knowledge and tools I’ve accumulated from previous work. Finally, I have a strong working relationship with this lab and its members. I know how to communicate and receive mentorship effectively. For all of these reasons, I expect this project to be my most fruitful and productive yet. 


\end{document}